%        File: revise.tex
%     Created: Wed Oct 09 02:00 PM 2013 P
% Last Change: Wed Oct 09 02:00 PM 2013 P
%

%
% Copyright 2007, 2008, 2009 Elsevier Ltd
%
% This file is part of the 'Elsarticle Bundle'.
% ---------------------------------------------
%
% It may be distributed under the conditions of the LaTeX Project Public
% License, either version 1.2 of this license or (at your option) any
% later version.  The latest version of this license is in
%    http://www.latex-project.org/lppl.txt
% and version 1.2 or later is part of all distributions of LaTeX
% version 1999/12/01 or later.
%
% The list of all files belonging to the 'Elsarticle Bundle' is
% given in the file `manifest.txt'.
%

% Template article for Elsevier's document class `elsarticle'
% with numbered style bibliographic references
% SP 2008/03/01
%
%
%
% $Id: elsarticle-template-num.tex 4 2009-10-24 08:22:58Z rishi $
%
%
%\documentclass[preprint,12pt]{elsarticle}
\documentclass[answers,12pt]{exam}

% \documentclass[preprint,review,12pt]{elsarticle}

% Use the options 1p,twocolumn; 3p; 3p,twocolumn; 5p; or 5p,twocolumn
% for a journal layout:
% \documentclass[final,1p,times]{elsarticle}
% \documentclass[final,1p,times,twocolumn]{elsarticle}
% \documentclass[final,3p,times]{elsarticle}
% \documentclass[final,3p,times,twocolumn]{elsarticle}
% \documentclass[final,5p,times]{elsarticle}
% \documentclass[final,5p,times,twocolumn]{elsarticle}

% if you use PostScript figures in your article
% use the graphics package for simple commands
% \usepackage{graphics}
% or use the graphicx package for more complicated commands
\usepackage{graphicx}
% or use the epsfig package if you prefer to use the old commands
% \usepackage{epsfig}

% The amssymb package provides various useful mathematical symbols
\usepackage{amssymb}
% The amsthm package provides extended theorem environments
% \usepackage{amsthm}
\usepackage{amsmath}

% The lineno packages adds line numbers. Start line numbering with
% \begin{linenumbers}, end it with \end{linenumbers}. Or switch it on
% for the whole article with \linenumbers after \end{frontmatter}.
\usepackage{lineno}

% I like to be in control
\usepackage{placeins}

% natbib.sty is loaded by default. However, natbib options can be
% provided with \biboptions{...} command. Following options are
% valid:

%   round  -  round parentheses are used (default)
%   square -  square brackets are used   [option]
%   curly  -  curly braces are used      {option}
%   angle  -  angle brackets are used    <option>
%   semicolon  -  multiple citations separated by semi-colon
%   colon  - same as semicolon, an earlier confusion
%   comma  -  separated by comma
%   numbers-  selects numerical citations
%   super  -  numerical citations as superscripts
%   sort   -  sorts multiple citations according to order in ref. list
%   sort&compress   -  like sort, but also compresses numerical citations
%   compress - compresses without sorting
%
% \biboptions{comma,round}

% \biboptions{}


% Katy Huff addtions
\usepackage{xspace}
\newcommand{\Cyclus}{\textsc{Cyclus}\xspace}%
\newcommand{\Cyder}{\textsc{Cyder}\xspace}%
\usepackage{color}
\DeclareMathOperator{\erf}{erf}
\DeclareMathOperator{\erfc}{erfc}
\usepackage[hyphens]{url}

%\journal{Advances in Engineering Software}

\begin{document}

%\begin{frontmatter}

   % Title, authors and addresses

   % use the tnoteref command within \title for footnotes;
   % use the tnotetext command for the associated footnote;
   % use the fnref command within \author or \address for footnotes;
   % use the fntext command for the associated footnote;
   % use the corref command within \author for corresponding author footnotes;
   % use the cortext command for the associated footnote;
   % use the ead command for the email address,
   % and the form \ead[url] for the home page:
   %
   % \title{Title\tnoteref{label1}}
   % \tnotetext[label1]{}
   % \author{Name\corref{cor1}\fnref{label2}}
   % \ead{email address}
   % \ead[url]{home page}
   % \fntext[label2]{}
   % \cortext[cor1]{}
   % \address{Address\fnref{label3}}
   % \fntext[label3]{}

\title{Rapid Methods for Radionuclide Contaminant Transport in Nuclear Fuel 
        Cycle Simulation}

   % use optional labels to link authors explicitly to addresses:
   % \author[label1,label2]{<author name>}
   % \address[label1]{<address>}
   % \address[label2]{<address>}


%\author[uiuc]{Kathryn Huff}
%        \ead{kdhuff@illinois.edu}
%  \address[uiuc]{Department of Nuclear, Plasma, and Radiological Engineering,
%        118 Talbot Laboratory, MC 234, Universicy of Illinois at
%        Urbana-Champaign, Urbana, IL 61801}
%
% \end{frontmatter}

\section*{Second Round Review General Response}
I would like to again thank the reviewers for their detailed assessment of 
this paper. 


\section*{Reviewer 2}
\begin{questions}
\question Minor note: Eq. 6 looks like you forgot to insert a line break in your equation array. (This is apparent in your LaTeX changelog.)
\begin{solution}
        Thank you for catching this. The line has been broken. 
\end{solution}

\question Line 219: Looks like there's an extra 1 in the mass subscript? 
($m_{1i}$)
\begin{solution}
        Thank you for catching this. The 1 has been removed.
\end{solution}

\question Line 391: Forgot to close parentheses
\begin{solution}
        Thank you for catching this. The closing parenthesis has been added.
\end{solution}

\question Line 397: Perhaps indicate *figures* 9 through 13.
\begin{solution}
        Thank you for catching this. The word ``Figures'' has been added.
\end{solution}

\question Lines 462-463: I think it would be more accurate to say, ``These 
isotopes have effectively no solubility limit'' rather than saying they are 
``infinitely soluble.''

\begin{solution}
        Thank you for catching this. This has been edited exactly as suggested. 
\end{solution}

\question URL for reference 6 points to the presentation, but the paper URL 
(which the page number corresponds to) is: 
\url{https://inis.iaea.org/search/search.aspx?orig_q=RN:44052619}
\begin{solution}
        Thank you for catching this. This has been edited exactly as suggested. 
\end{solution}



\section*{Reviewer 3}

\question I know it takes a lot of work to revise a paper. The revisions were very thorough, and this is a good paper. I am pleased to recommend this paper to be published. 
\begin{solution}
        Thank you.
\end{solution}


\section*{Reviewer 4}

\question It would have been helpful to include a real-life application of the method, which might also have helped improve the methods you used.  This is an excellent paper however
\begin{solution}
Thank you. I hope to present a real-life application of this method, beyond the
simplified homogeneous clay-like medium presented here, in a future paper.
\end{solution}


\noindent\makebox[\linewidth]{\rule{\paperwidth}{0.4pt}}
Previous review comments and their responses are retained below for
reference.
\noindent\makebox[\linewidth]{\rule{\paperwidth}{0.4pt}}



\section*{First Round Review : General Comments}
I'd like to extend my immense thanks to the reviewers, whom I think have
improved this manuscript dramatically with their insightful comments. Please
see below for discussion of each.  

Differences between the submitted version and this revision can be found at the 
following url: \url{https://goo.gl/qRQuXn}. To see line changes, click on the 
``files changed'' tab.

To view a pdf of these review comments, see the following url:
\url{https://github.com/katyhuff/2017-huff-rapid/raw/master/revise/revise.pdf}

\section*{Reviewer 1}

\question I enjoyed reviewing this manuscript, but I found it difficult to 
        understand.
The manuscript seemed to lack specific goals and conclusions. It was very
abstract and long.  This reviewer wondered if the addition of examples would
have made the manuscript more useful. The title used the word ``rapid,'' but 
        this
property was not developed in the text. Was this a review paper? Some of the
figures in the current manuscript were illegible. I hope that the following
comments can improve the manuscript.
\begin{solution}
Thank you for your particularly detailed comments. I have edited the manuscript
        once through for brevity.  Specific goals (to demonstrate the 
        capabilities of the Cyder software in comparison to GoldSim) and 
        conclusions (appropriate validity for medium-fidelity fuel cycle 
        simulations) were added to the introduction.  Figures and figure 
        captions were reviewed and many were enlarged for better legibility. 
        All will be provided to the journal office in high resolution.
\end{solution}

\question This abstract seemed to lack a clear statement of the goals of the 
        paper, the methods used, results, and conclusions.
\begin{solution}
Thank you for this comment. I have re-written the abstract to include concrete 
details of the purpose, models, results, and conclusions.  \end{solution}

\question Page 2 : Somewhere in this paragraph, we needed a statement about the 
        goals and objectives of this study.
\begin{solution}
Goals and objectives have been added. Specifically, to describe the design, 
        development, and mathematical models within Cyder and to verify the 
        models in Cyder against a higher fidelity DOE tool, the Clay GDSM.
\end{solution}


\question Line 35. Was this section still part of the introduction?

\begin{solution}
It's not intended to be. This section begins the first task of this paper, 
        which is to describe the models implemented as part of the work. This 
        is the software-paper equivalent of the methods section.
\end{solution}

\question Figure 1. Was z depth?
\begin{solution}
Yes, I have now added this fact to the caption.
\end{solution}

\question Page 4 : Line 62. What were the outer and inner components?

\begin{solution}
        The example intended to be general, but I see that this is a confusing 
approach. I have made this more explicit. The example now specifically refers 
to the waste form and waste package as the inner and outer components.  
\end{solution}
 

\question Figure 2. All abbreviations in figures need to be defined in the 
        figure caption.
\begin{solution}
Great catch. Done for Figs 2, 3, and 4.
\end{solution}
 

\question Page 5: What did ``10[kg]'' mean?
\begin{solution}
        10kg was the example source term (mass of radionuclides in the waste 
        form).
\end{solution}
 

\question Line 82 plus. An example would have been helpful.
\begin{solution}
This and all other subsections of section 2.1 reference the overarching example 
        explained in its parent section. This example will be carried through 
        the whole timestepping section. I have added text to communicate this 
        better to the reader. 
\end{solution}

 

\question Page 6 : Equation 4. Why 10?
\begin{solution}
        10 kg was the example source term (mass of radionuclides in the waste 
        form).
\end{solution}

\question Page 7 Lines 115 to 128. An example would have been helpful. This 
manuscript was abstract, and not easy to read.

\begin{solution}
This and all other subsections of section 2.1 reference the overarching example 
        explained in its parent section. This example will be carried through 
        the whole timestepping section. I have added text to communicate this 
        better to the reader. 
\end{solution}
 

\question Line 130. This statement needs a reference.

\begin{solution}
This is a very good point. Barrier materials do not degrade appreciably over 
short time scales, so I have rephrased this completely to recognize the general 
stability of most materials in a repository environment over long time scales.  
\end{solution}

\question Figure 5. Again, we need to define all figures symbols and 
abbreviations in the figure caption.
\begin{solution}
Great catch. I have added the missing definition for $V_T$.  \end{solution}

\question Page 9 Figure 6 caption. Why is only Vdf available for transport?
\begin{solution}
Great question. This particular model is naive. It assumes the portion of the 
        component that remains solid  has a negligible solubility. This model 
        is appropriate for very durable components, such as borosillicate 
        glass.  For this model (but not for all four) the dissolution of the 
        component material is handled in the degradation rate parameter rather 
        than in a matrix solubility parameter. A better model would include a 
        parameter for matrix solubility as well. I believe this should be a 
        future feature in the package. At the moment, users interested in this 
        higher fidelity can choose the 1D PPM model instead. 
\end{solution}

\question Equation 12. What did ``degraded volume'' mean?
\begin{solution}
This is the volume that reflects the degradation state of the component. The 
        volume is the degradation rate times the initial volume. I have called 
        out $V_d$ in the text to clarify this.
\end{solution}

\question Page 10 Equation 15. What was the ``degraded solid volume?''
\begin{solution}
        This is the volume that has been degraded, but has not dissolved into 
        the mobile fluid. I have added text to clarify
        $V_{is}$,  $V_{ds}$, $V_{if}$, and
        $V_{df}$.  \end{solution}
 

\question Page 11 Line 173. ``Into'' or ``onto?''
\begin{solution}
Great catch. I did mean onto. Fixed.
\end{solution}

\question No line number. The ``solid concentration?'' What did that mean?
\begin{solution}
This is the concentration of the contaminant present in the solid component 
        matrix volume. This has been further clarified in the text.
\end{solution}
 

\question Line 177. What type of degradation?
\begin{solution}
        I have added text to explain that the model is agnostic to the 
        mechanism of degradation, as it is simulated based only on the rate.  
        Release based on this degradation is congruent.
\end{solution}

 

\question No line number. What did ``sorbate is in the degradable solids'' 
mean?
\begin{solution}
I meant sorbed material. Since 'sorbate' was unclear, I have replace the use of 
        sorbate with 'sorbed material' instead. Thank you.
\end{solution}
 

\question Page 14 Line 210 should begin with ``Because.'' Since is a time such 
as ``Since this morning, I've been reading this manuscript.''
\begin{solution}
I chuckled. Thanks for that. I have fixed this, and noted the lesson for the 
future.  \end{solution}

 

\question Page 22 Line 330. Were these experiments done for the current study? 
How much of the current manuscript is new material, and how much was already 
published in reference 21?
\begin{solution}
Excellent question. You were not the only reviewer to find this unclear. All 
        simulations were indeed, done specifically for this work and by this 
        author. Indeed the Clay GDSM results (figures 14, 16, 18, and 19) were 
        discussed in reference 21 (notably, this was just a 4-page ANS summary 
        and not an archival journal article).
        I have now clarified this distinction in the captions of figures 14, 
        16, 18, and 19. Those simulations were used as reference, as a higher 
        fidelity verification for the simulations discussed in the section. The 
        simulations discussed in this section, conducted with Cyder, are the 
focus of all other figures and discussion.  \end{solution}

 

\question Line 342. ``Were conducted'' Why is this statement in the results 
section?
\begin{solution}
I'm not sure I understand the concern here. I tend to follow a style guide in 
        which the work presented in the paper is discussed in the past tense, 
        particularly in the results section. I think that the reviewer's 
        comment stems from this past tense being confusing. I have clarified 
        that sentence to state, instead, ``In the present work, many numerical 
        experiments successfully verified the capabilities of the Cyder 
        software library''. I hope this form of the sentence is more clear.
\end{solution}

 

\question Line 344. What was a ``global parameter''?
\begin{solution}
I just meant universal parameters, throughout the simulation. So, I have 
        changed this text to say ``simulation parameters''.
\end{solution}

 

\question Line 348. What did ``accepting 1 waste stream'' mean?
\begin{solution}
        I have changed this text to say ``No more than one waste stream object 
        is stored per waste form.'' I hope that is more clear. I just mean 
        that, in the object oriented sense, the waste form object (e.g. a 
        certain size and shape of borosilicate glass) is associated with one 
        waste stream object (the isotope vector of waste material, such as 1kg 
        of Tc99 or 2kg of Cs137, or some combination of many masses of 
        isotopes.).
\end{solution}

 

\question Line 351. What did ``a far field component'' mean?
\begin{solution}
I added a parenthetical to be explicit. It now says ``a far field component 
        (i.e. the host rock)''.
\end{solution}

 

\question Line 358. A solubility limitation was set? Does this mean that CYDER 
does not use actual solubility values?
\begin{solution}
The user must provide the solubility values. While this provides the user with 
        complete control, it may be inconvenient for some users. Future 
        extensions to \Cyder will include a default database, ideally through 
        the PyNE toolkit. I have added text to this effect.
\end{solution}

 

\question Figure 9. We need to define all symbols and abbreviations used in a 
figure.
\begin{solution}
Good catch. Done.
\end{solution}

 

\question Figures 10, 11, 12, and 13 are illegible. They are too small. Label 
all axes.
\begin{solution}
The sizes of each has been increased twofold. Axes are all labelled.  
\end{solution}

\question Page 24 Line 389 hints some of the material in the current manuscript 
has already been published in reference 21.
\begin{solution}
        Reference 21 was an ANS conference summary which presents results of 
        the GoldSim-based Clay GDSM model from DOE. The present work uses that 
        previous work as a benchmark for comparison. That is, in the present 
        work, results from Cyder are compared to those Clay GDSM model results 
        from the reference. Key plots are reproduced here. To make this 
        distinction more clear, I have added citations to all plot captions 
        which I produced as a part of the previous work.
\end{solution}


\question Line 399. What is a ``sharp turnover?''
\begin{solution}
I just meant the elbow in the figure. I have now changed this to be more 
        explicit about the meaning. It now says `` Specifically, in Figure 15, 
        marked transition to the solubility-limited regime is seen where the 
        solubility limit exceeds the point at which it limits movement.  
\end{solution}

 

\question Figure 14 is illegible.
\begin{solution}
I have increased the size of Figure 14.
        It will be provided separately to the journal in high resolution. 

\end{solution}

 

\question Page 26 Line 419. Kd has units.
\begin{solution}
That is very much true. However, these results are referring to a 
multiplication factor applied to $K_d$. I have called this the $K_d$ factor. 
While $K_d$ varies by isotope, the factor was applied to all isotope $K_d$ 
values.  \end{solution}


\question Figure 16 is illegible.
\begin{solution}
I have increased the size of Figure 16.
        It will be provided separately to the journal in high resolution.  
\end{solution}

 

\question Lines 432, 434, and 436. Try to use the words "high" and "low" for 
vertical references rather than to qualify concentrations or properties.
\begin{solution}
Great advice. I have changed the text to use fast, slow, and large instead.
\end{solution}

 

\question Figures 18 and 19 are illegible.
\begin{solution}
I have increased the size of Figures 18 and 19.
        They will be provided separately to the journal in high resolution.  
\end{solution}

 

\question Line 336. Are these conclusions?
\begin{solution}
Thank you for pointing this out. You are quite right that the Results and 
        Discussion Section is no replacement for concise conclusions. I have 
        renamed the final subsection (previously called significance). It is now 
        a section of its own entitled ``Conclusions.'' I have not changed the 
text of that section, however.  \end{solution}

 

\question Line 447. ``Rapid'' relative to what? Compared to what?
\begin{solution}
        This is a good point. I neglected to describe (beyond the motivation in 
        the beginning) the speed up provided by Cyder (over GDSM). I have now 
        added a remark on their comparison in the results section and have made 
        the magnitude of the speed up more clear there. (Cyder performs 
simulations in minutes where the higher fidelity GDSM requires hours.) 
\end{solution}

 

\question Page 29 Lines 456 to 472. The ``conclusions`` only seemed to promote 
the use of CYDER. There were no specific conclusions. What was new in this 
study?  \begin{solution}
Point taken. I have reworked the conclusion to focus on the facts. The design 
        and development of Cyder itself is the primary effort being described 
        in this paper, so a summary of Cyder's capabilities remains in the 
        conclusion. However, I have framed this discussion to also emphasize 
        the success of the Cyder verification effort that was the related focus 
        of this work.
\end{solution}


\section*{Reviewer 2}

\question This article describes the implementation of a medium-fidelity 
contaminant transport model applicable to the Cyclus nuclear fuel cycle 
simulator. Overall this fills an important gap in fuel cycle simulation tools 
for back-end nuclear fuel cycle assessment.

Overall this article is well-written and well-organized, with careful attention 
given to methodological details used within the model. One minor issue I have 
though is that given the specialized nature of the discussion (i.e., hydrology 
and contaminant transport), there are a number of areas where I feel the 
clarity of the manuscript could be improved by simply being more explicit with 
nomenclature used. For example, in line 181, where is the relationship between 
$m_ds$ and $m_T$ outlined? Similarly, what does "d" (used in Eq. 30) denote - 
bulk density? Even with some background in the subject matter (which I cannot 
presume all readers to have) I found this difficult to follow without further 
explanation of nomenclature.
\begin{solution}
Thank you for your kind words. In response to this comment, as well as other 
        similar comments by yourself and other reviewers, I have added 
        additional notes in the text to call out  definitions of subscripts 
        such as $ds$ (degraded solid) and $T$ (total). Relationships were 
        previously described in equations ($m_{ds}$ in Eq. 22), but I failed to 
give english descriptions of these. I have now added english definitions of 
such volume and mass subscripts to make this more clear. Thanks!  
\end{solution}

\question Similarly, in Figures 9-13, what does $F_d$ refer to? I assume from 
context that $S_{ref}$ refers to the solubility limit imposed for the nuclide, 
but again - please make your nomenclature more explicit.  \begin{solution}
        This is a great catch. I never defined $F_d$ (fractional degradation 
rate). You are correct about $S_{ref}$ I have now added a definition of both 
$F_d$ and $S_{ref}$ to all figure captions where they are used.  \end{solution}

\question Likewise, given that the purpose of this study is a qualitative 
comparison of the behaviors of the Cyder generic repository model to a prior 
DOE-developed model (the Clay GDSM), it might be helpful to identify which 
figures originate from the latter. For example, are Figures 14, 18, and 19 from 
the Clay GDSM? If so, please indicate this explicitly in the captions.  
\begin{solution}
Correct. This has been fixed with citations in the captions. I think all 
reviewers made this comment!  \end{solution}

\question With respect to Equations 5 and 6, if $m_{ij}$ denotes the mass flux, 
then shouldn't Equation 6 specify $m_{ij}$ as a time-integrated quantity, given 
that $m_j(t_n)$ is a mass quantity and $m_{ij}$ is a mass flux?
\begin{solution}
        Good catch. Fixed.
\end{solution}

\question In equation 28, $c_p$ should be a capital C.
\begin{solution}
Correct! Fixed.
\end{solution}

\question Line 240 (``The correspon'') appears to be a typo.
\begin{solution}
Thanks! Fixed.
\end{solution}

\question In section 3.2.2 (lines 407-408), it would be helpful to define the 
relationship of the retardation factor $R_f$ to $K_d$. i.e., $R_i = 1 + \rho_b 
* K_d / \theta$, where $\rho_b$ is the bulk density and theta is the porosity.  
Similarly, this is made most clear to the reader if you then illustrate the 
effect of the retardation factor in terms of the effective contaminant 
velocity, i.e. $v_{eff} = v/R$, where v is the average linear velocity.
\begin{solution}
        This is an excellent point and I have added the relationship between 
        $R_f$ and $K_d$ in the text as recommended.
\end{solution}

\question I find the comparison of Figure 20 to Figures 18 \& 19 somewhat 
confusing; the latter two use a log-log scale, yet Figure 20 is on a linear 
scale. Is all we are supposed to see a saturation behavior from degradation 
rate? I still think this comparison would be better expressed as a log-log 
comparison.
\begin{solution}
I agree that the comparison would be clearer if they were both log-log. 
        Unfortunately I did not have time to reproduce these graphs (I only had 
        two days with your detailed comments). However, I hope to be able to 
        make this change before publication if I am given that opportunity.
\end{solution}

\question In both the introduction and the conclusions, you indicate that Cyder 
is compatible with Cyclus (specifically version 0.3). What is the status of its 
compatibility with the most recent Cyclus v1.5 release? Are there plans to 
maintain compatibility if it is not presently compatible?
\begin{solution}
Quite correct! This work was conducted with Cyclus v.0.3. Current work is 
        ongoing to bring Cyder up to date with the most recent version of 
        Cyclus. This work will serve as fodder for integration testing in that 
        transition, ensuring that the same results are achieved with the new 
        version of Cyder. I have added text to the paper indicating this.
\end{solution}

\question Finally, with respect to citations: Is there a more complete, 
locatable citation for [1] and [5]? Citation [6] is incomplete: the full 
citation is:

Boucher, L., Alvarez Velarde, F., Gonzalez, E., Dixon, B.W., Edwards, G., Dick, 
G., \& Ono, K. (2012). International comparison for transition scenario codes 
involving COSI, DESAE, EVOLCODE, FAMILY and VISION. Proceedings of the Eleventh 
Information Exchange Meeting on Actinide and Fission Product Partitioning and 
Transmutation, (p. 406). Nuclear Energy Agency of the OECD (NEA)

ISBN: 978-92-64-99174-3
URL: \url{https://inis.iaea.org/search/search.aspx?orig_q=RN:44052619}
\begin{solution}
All fixed. URLs added, bibliography entry for Boucher et al. fleshed out.
\end{solution}

\question Is citation [8] a thesis? Please make this more explicit and indicate 
any kind of URL or other locator device, if possible. Also, should citation 
[10] point to http://fuelcycle.org? Finally, citation [19] clearly appears to 
be in error; please check your BibTeX database for this one.
\begin{solution}
        Yes, 8 was a thesis, 10 now points to fuelcycle.org (an alias of 
        cyclus.github.com), and the URL for 19 has been updated to avoid 
        library proxy servers.
\end{solution}

\section*{Reviewer 3}

\question   Revisions are more moderate than minor, but should not be overly 
difficult
to complete. Please see the attachment. One of the main concerns was that since
this a software journal, some of the math that is well known to researchers in
fate and transport might not be known to readers of this journal. This is
discussed more in the review.
\begin{solution}
Thank you for your very detailed comments. Your main concern is well taken. 
Other reviewers (above) made specific requests regarding this balance between 
software and method.  In addressing those, I believe the paper has become much 
more accessible to all readers. Examples have been made more explicit, I have 
made attempts to reduced the wordiness and abstraction of the paper somewhat. I 
hope these changes have made the mathematical discussion more accessible to a 
broad audience.  \emph{Please note that I did not have access to your attachment for 
quite a while (for some reason, it was only available to the editor in 
evise). After phone calls and emails, it was sent to me, but 
only two days before the revision submission deadline. I hope you will forgive 
that I was rushed somewhat when responding to your comments. I very much 
        appreciate the detail with which you reviewed this paper.} 
\end{solution}

\question
Overall,  this  is  a  great  paper,  well  structured  in  discussing  the  
CYDER  library,  then  the  various  means  of calculating  radionuclide  
transport  behavior,  then  the  underlying  mathematical  models,  and  
finally  the  results and  discussion.  At  the  beginning  the  paper  would  
benefit  from  clarifying  what  is  meant  by  ``technologically coupled.''  
Placing  this  paper  in  the  context  a  greater  discussion  of  why  
nuclear  waste  disposal  decisions  rely on  the  full  nuclear  fuel  cycle  
in  the  abstract  and  introduction  would  make  the  paper stronger.  It 
would help to  provide  what  a  ``medium-fidelity''  time  step  is  in  terms  
of  hydrologic  modeling  of  a  nuclear  waste repository.  Because  there  is  
a  significant  amount  of  mathematical  formulations  in  this  paper,  
describing  the physical  conditions  is  of  high  importance.  Please  review  
the  figures  and  consider  adding  more  descriptive content. More comments 
on the figures is discussed below.  \begin{solution}
Thank you.
        Regarding technological coupling, I have added a sentence in the 
        introduction explaining that fuel cycles and spent fuel repositories 
        are coupled by SNF volume, isotopic composition, mass, disposition, 
        activity and other variables. These variables depend on the fuel cycle 
        choices made upstream and impact repository capacity, loading strategy, 
        and performance.  Regarding the time step for medium fidelity, I should 
        note here first that the reference to fidelity is not specifically with 
        regard to time, but also with spatial and physical simplifications, 
        depending on the model (among the four) chosen for each component. In 
        the timestepping section, I have added a note about the default time 
        step size (1 month) and its appropriateness for this level of fidelity 
        on disposal timescales.  However, the user is welcome to use 
        arbitrarily small or large timesteps.  Regarding figures and content, I 
        have made changes according to your comments below.
\end{solution} 

\question The  vertical  depth  is  mentioned  first  on  page  2  line  42,  
but  not  again  until  page  16,  equation  (43),  and  then page  18,  line  
263.  Typically,  for  modeling  radionuclide  transport  in  the repository, 
flow would be considered in  both  the  +x  and  +z  directions as indicated in 
Figure 1. (Some models also consider matrix diffusion in the +z  direction).  
It  is  not  stated  in  the  paper  why  only  the  flow  in  the  z  
direction  is  considered  for this model.  For  a  repository  located  at  a  
500  m  depth,  for  example, why would the flow in the z direction be 
significant?  There  are reasonable assumptions that can be made, but this is 
why is it so important to describe precisely the physical conditions that this 
model describes in order to justify this.  \begin{solution}
I have added text on page 18 to explain that model is simply making a 1-D 
        transport assumption for simplicity. The vertical direction was chosen
        because it is the most conservative choice, among directions, since it 
        is the shortest path to the biosphere (and therefore, perhaps, humans).  
        As you note, flow is not significant in any direction for geologic 
        media under consideration for repository sites internationally. The far 
        field is an excellent barrier to release and of course repositories are 
        accordingly quite good at containment.
\end{solution} 

 
\question In  Figures  9-13,  what  information  exactly  is  being  plotted  
on  the  graphs should be clearly stated. Section 3.1 generally  describes  the  
simulations, but not what is shown on each figure. The captions should be 
descriptive, even if this means there are only one or two figures per page.  
\begin{solution}
Thank you for this comment. I  have added more detail to these captions to 
        better define all symbols that appear to make more clear the meaning of 
        the plots.
\end{solution} 
 
\question Page  2  line  7-18:  Explaining  why  dynamic  waste  disposal  
calculations  are  valuable,  as  opposed  to  the static outputs from other 
tools, would make the paper stronger.  \begin{solution}
        This has been done (see previous comment in response to your initial 
        mention of this issue.)
\end{solution} 

\question Page 2 line 21:  Provide an example of what kind of time step `medium 
fidelity models' includes \begin{solution}
        This has been done (see previous comment in response to your previous 
        mention of this issue.)
\end{solution} 
 
\question Page  2  line  24:  Explain  the  difference  between  using  Cyder  
as  a  standalone  library  and  in  collaboration  with Cyclus 
\begin{solution}
I have explained this statement by emphasizing that it is compiled as an 
        independent shared object library with its own application programming 
        interface. This terminology will be appropriate for the readers of this 
        journal.
\end{solution} 
 
\question Page  4  lines  59-61:  ``These  calculations  proceed  from  the  
inner-most  component  to  the  outermost component,  with  mass  transfer  
calculations  conducted  at  the  boundaries.'' Could  use  a  labeled  diagram 
that  describes  what  the  inner  components  are  versus  the  outer  
components  or  provide  more  detail  in Figure 1. I cannot tell what is the 
``geosphere'' and what are the various containment layers.  \begin{solution}
You are not the only reviewer to feel this way, and I appreciate that you have 
        pointed out how confusing this is. I have elected to add additional 
        detail to the caption of Figure 1 to address this. It now says ``In 
        \Cyder, as in a canonical drift-tunnel repository, waste form 
        components (the innermost components) are contained by waste package 
        components which are, in turn, emplaced in a buffer component (the 
        backfilled emplacement tunnel into which waste packages are loaded). 
        That buffer component contains many other waste packages, spaced evenly 
        in a horizontal grid.  The geosphere (the outermost component) occupies 
        all space below the repository surface and outside of the buffer 
        components (emplacement tunnels). The \Cyder repository layout has a 
        depth ($\Delta z$) and package spacing defined by the user input 
        ($\Delta x$ within the drifts and $\Delta y$ between drifts.''

\end{solution} 
 
\question Page  4  line  74:  Please  explain  source  and  sink  more  
thoroughly  for  non-Cyclus  users,  and include more description of these in 
Figure 2.  \begin{solution}
Excellent observation. These are actually unrelated to the \Cyclus notion of 
        the source and sink. So an explanation may be even more important for 
        Cyclus users (for whom the term is overloaded)! Thanks for pointing 
        this out. I have added a sentence to clarify the meaning of these two 
        components in the example.
\end{solution} 
 
\question Page 5, Figure 4: The asterisk for $t^*_n$ is not defined 
\begin{solution}
I have added, to the caption, more detail about this timestep updating syntax 
and all other symbols used in this figure.  \end{solution} 


\question Page  6,  Equation  5: While   is standard notation for mass flow 
rate, it might be helpful to clearly define it in the text \begin{solution}
This equation has been clarified with a statement in the text as well as an 
        additional equation 6 describing the integral relationship between $m$ 
        and $\dot{m}$.
\end{solution} 
 
\question Page  7-16:  Might  be  more  clear  to  include  a  table  that  
shows  the  differences  between  the  four Radionuclide  Mass  Balance  
models;  include  the  inputs,  the  built  in  assumptions,  the  strengths  
and  the weaknesses.  \begin{solution}
This is an excellent idea. I sketched a bit of this table idea to see it could 
        be done clearly, but I'm afraid I came up short. The table quickly 
        became unweildy, given the detail necessary to differentiate the models 
        in their many facets. I like this suggestion, but was unable to make it 
        work.
\end{solution} 
 
\question Page  8,  Equation  9:  $m_{jk}$  should be defined right below 
Equation 9 where it is introduced first as opposed to below Equation 10.  
\begin{solution}
Fixed.
\end{solution} 
 
\question Page  9, line 148: Would readers of this journal know the Dirichlet 
boundary condition? Perhaps include a description of it.  \begin{solution}
Good question. It is my understanding that readers of this journal are 
                engineers who write research software for engineering. After 
                looking at some examples among previous articles in this 
                journal, I think this is something that the reader will 
                understand, particularly since the form of the boundary 
                condition is given in the figure.
\end{solution} 
 
\question Page  9,  line  150-151:  ``is  the  average  concentration 
throughout the degraded volume'' at which point (j or k), or at the boundary 
between the two? The wording is a bit confusing.  \begin{solution}
        Good catch. I intended to refer to $C(r_j)$, the concentration at 
        $r_j$, which is the boundary between $j$ \& $k$.
\end{solution} 
 
\question Page 9, Figure 6: This figure would be clearer if $V_T$ was shown as 
full of the various constituent parts.  \begin{solution}
This seemed like a great idea, so I tried it (see below). However, I'm not 
        convinced it isn't actually more confusing, so I have chosen not to 
        make this
change. If the reviewer feels strongly that the below image is more clear, I
can certainly make this change, but I do feel like it may actually add to the 
        confusion.
\end{solution} 


\begin{figure}[h!]
  \begin{center}
    \def\svgwidth{\columnwidth}
    \input{./volumes.eps_tex}
  \end{center}
  \caption[Constituents of a Mixed Cell Control Volume]{The degraded volume is
  modeled as a degraded solid volume, $V_{ds}$, and a degraded fluid volume,
  $V_{df}$. The intact volume is modeled as an intact solid volume, $V_{is}$, 
        and
  an intact fluid volume $V_{if}$.  Only contaminants in $V_{df}$ are available
  for transport.}
  \label{fig:deg_sorb_volumes}
\end{figure}

\noindent\rule{\textwidth}{0.4pt}

\question Pages 9-10, equations 13-21: $\theta$  is not defined.  
\begin{solution}
The porosity, $\theta$, is defined in the paragraph immediately preceeding this 
        set of equations.
\end{solution} 
 
\question Page  10,  line  161:  Define  contaminant  masses  and  why  they  
are  the  ones  available  to  adjacent  components.  \begin{solution}
Thank you for this comment. I didn't realize how unclear this was. All masses
are contaminant masses in this context. I have added a reminder of this to the
beginning of the section at what was previously line 114.
\end{solution} 
 
\question Page  10,  line  165:  The  use  of the word 'available' is not 
really mathematically based. Because there are so many equations, the mass 
transfer of the contaminants should be precisely described.  \begin{solution}
Very true. I have replaced this with the more explicit wording, ``mass $m_{ij}$
that can be transferred to the adjacent component in the mass transfer phase.''
\end{solution} 
 
\question Page 12, Equations 30 \& 31: It is not clear how the result in 31 was 
obtained from 30.  \begin{solution}
Great catch. This was based on the definition of $V_{df}$ as a function of
$V_T$. I have added text to this effect. But, more importantly, as you noted,
there was an extra d. Thanks for noticing this level of detail. I have double
checked the source code and you will be pleased (as I was) to know that in the
implementation, the $V_{df}$ definiton is plugged in directly, so this typo is
in the paper only and did not appear in the code itself.  \end{solution} 
 
\question Page 15, lines 220-27: Similar to the comment on the Dirichlet 
condition, both the Cauchy and Neumann conditions should be clearly defined. 
They are used in Figure 8 but not clearly indicated either.  \begin{solution}
I have called these out by name in equations (40) and (41). Given the
engineering computation background of the readership, I suspect an elementary
explanation of Cauchy and Neuman conditions would underestimate the readers.
I hope that common differential equation boundary conditions of this type are 
        well
understood by engineers, particularly the computationally trained readers of
this journal.
\end{solution}
 
\question Page  16,  Equation  43:  Based  on  experience,  D  and  v  should  
be  diffusion coefficient (or dispersion) and advection,  respectively,  
although  they  are  not  defined  here.  Would  the  readers  of  this  
journal inherently understand  this,  however?  It  is  typically  just  good  
form  to  reference  all  the  variables  in  an  equation,  or state that D 
and v were defined previously in equation (x).  
\begin{solution}
        You are quite correct, I have moved the definitions of D and v to their 
        first mention. I do believe that the readers of this journal are 
        engineers and engineers are typically familiar with flow (so advection and diffusion kernels will 
        be familiar). However you are correct that the explanation needs to be 
        pointed out in this first mention. I have referenced the discussion of 
        advection and dispersion (which appears in the following mass transfer 
        section) from this section in which they first raise their 
        heads. Thank you for this suggestion.
        
\end{solution} 
 
\question Page  16,  Equation  44:  The  retardation  factor  is  not 
physically defined. This is an important parameter in radionuclide transport 
studies.  \begin{solution}
A definition has been added to its first mention, as a function of bulk 
        density.
\end{solution} 
 
\question Page 16, line 240: This line is cut off and sentence starting 'The 
correspon' does not make sense.  \begin{solution}
This has been fixed, see previous comments to reviewer 1.
\end{solution} 
 
\question Page  16,  line  246-49:  If  there  are  results  comparing  the  
explicit  and  implicit  modes  of  mass  transfer,  might  be  worth  
mentioning  where  they  are  discussed  below.  If  there  are  not  results,  
might  be  worth discussing the general success of the implicit mode.  
\begin{solution}
Any difference introduced by the explicitness vs. implicitness of the modes is 
        dwarfed by the differences between the models themselves. Nonetheless, 
        in the results section, I have added a reference to the much longer 
        dissertation where such additional results can be found. 
\end{solution} 
 
\question Page  16,  lines  250 \& 251: Advection and dispersion have been 
mentioned enough to warrant a definition at  its  first  mention  in  the  
paper.  It  is  defined  on  page  17,  line  253,  but  discussed numerous 
times in the paper already.  
\begin{solution}
This has been solved on page 16 by referencing this definition at first 
        mention. 
\end{solution} 
 
\question Page 16, line 251: 'equationfor' needs a space inserted.  
\begin{solution}
Fixed.
\end{solution} 
 
\question Page  20,  Line  299:  Equation (44) is mentioned as an afterthought 
as to imply that the reader should have known  that  (43)-(44)  was  what  
C(z,t)  represents  in  this  section.  Because  there  is  a  lot  of  
derivations,  an earlier mention to this in Section 2.3 should be included.  
\begin{solution}
        I'm sorry to say that I don't understand this comment. Section 2.3 is 
        after, not before, (43) - (44). If you mean section 2.2.3, then please 
        note that each of the four subsections (2.2.1, 2.2.2, 2.2.3, and 2.2.4) 
        seek to describe \emph{different} models for determining C(z,t). So, in 
        fact, equation (44) only applies to the model discussed in 2.2.4 and is 
        not at all related to the model in 2.2.3.
\end{solution} 
 
\question Page 23, figure 9: Unclear what the CompID values represent, do not 
know what $F_d$ represents.  \begin{solution}
These two items have now been defined in the captions of all of the figures 
that use them. See response to reviewer 1.  \end{solution} 
 
\question Page  23,  figures  10-13:  Unclear  why  the  IsoID  legend  is  
present  as  the  graphs  only  appear  to  include  $^{238} U$.  
\begin{solution}
The graphs include all isotopes listed. It is simply the case that the vast 
        majority of the material in the example was made of $^{238}U$ isotopes.  
        So, blue is the predominent color, and all others are dwarfed at the 
        bottom of the plot.
\end{solution} 
 
\question Page 24, line 372: 'accross' is spelled wrong \begin{solution}
Fixed.
\end{solution} 
 
\question Page  26,  Figure  16:  Both  $^{129}I$  and  $^{79}Se$  exhibit  a  
markedly  distinct  behavior  than  the  other  radionuclides and  should  be  
explained  as  to  why.  Additionally,  because  only  $K_D$ is shown on the 
graph, but in fact the analysis  is  focused  on  the  retardation  
coefficient,  the  relationship  between  the  two  should  be  shown.  
Additionally,  the  effects  of  the  retardation  factor  and  $K_D$  are 
dependent on the sorption isotherm that is assumed. This should be included in 
the discussion.  
\begin{solution}
This is correct, I have pointed out that the different behavior of these 
        nuclides is due to their near infinite solubility. Additionally, I have 
        now added a definition of the relationship between $K_d$ and $R_f$ at 
        the very beginning of the section. 
\end{solution} 
 

\question Page 28, Figure 18, 19: The text on the graphs is really small and 
hard to read unless one zooms to 200\%.  While  this  isn't  really  that  much  
of  an  issue,  it  might  be  prudent  to  enhance  the  graphs  if  possible,  
just fyi.  \begin{solution}
I have increased the size of Figures 18 and 19.
        These will be provided separately to the journal in high resolution.  
\end{solution} 

\section*{Reviewer 4}

\question  Some aspects of material transport are treated somewhat 
superficially. The
first is that the radionuclide waste material that would be transported by
groundwater is essentially insoluble uranium dioxide to the surface of which
fission products are bound by adsorption, which is essentially a weak chemical
bond.  This suspension of particles in water is treated as if it were a water
solution rather than a suspension.  Relatively straightforward laboratory
experiments investigating the difference in transport between a true solution
and a suspension might have resulted in somewhat different parameter values.
Alternatively, discussion of such experiments might be found in the literature.
The Hanford site provides a physical example of contaminated groundwater (from
the French drains) moving through an adsorbing matrix, and might have been
cited. In addition, published data of groundwater movement at the proposed
repository site in Nevada might have been cited.  Some tabulation of Cyder
results with results from some of the cited models would be illuminating.  
\begin{solution}
Thank you for these suggestions. It is true that some of the models treat 
        transport very superficially. The software implementation aimed for 
        speed, rather than high fidelity. I hope that future improvements can 
        emphasize fidelity without sacrificing speed. Regarding the treatment 
        of particle suspension as solution rather than solution, you're quite 
        right that this is a weakness. Your suggestion for including data 
        retrieved from Handford or Yucca Mountain would be excellent future 
        feature improvements to this framework. The current implementation made 
        comparisons with simulations of a homogenous medium, reducing 
        chemistry, generic clay host rock environment. Given the renewed 
        interest in Yucca Mountain, verification against high fidelity tuff 
        models is very desirable.  The comparisons between Cyder 
        and the cited GDSM model were displayed as plots, rather than tables, 
        due to the sheer size of the tables that would be needed. Importantly, 
        the raw data for the Cyder results can be retrieved in its open source 
        repository. Thank you so much for these suggestions. 
\end{solution}

\end{questions}
  \end{document}

  %
  % End of file `elsarticle-template-num.tex'.


