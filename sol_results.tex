
The above multi-component simulation was conducted accross a range of 
reference solubility limits. This parametric analysis was conducted to show that, for an 
arbitrary isotope, the expected solubility limitation
behavior is captured. In the case of real isotopes in a full simulation, the 
same model will be invoked with real parameters for each isotope. Thus, the 
this model agreement is representative in all cases.

The results acheived with \Cyder were compared to the results of a parametric sensitivity 
analysis reported in \cite{huff_key_2012}. That analysis, conduced with a more 
detailed generic radionuclide transport model, showed that for solubility 
limits below a certain threshold, the dose releases were directly proportional 
to the solubility limit, indicating that the radionuclide concentration 
saturated the groundwater up to the solubility limit near the waste form.  For 
solubility limits above the threshold, however, further increase to the limit 
had no effect on the peak dose. This demonstrates the situation in which the 
solubility limit is so high that even complete dissolution of the waste 
inventory into the pore water is insufficient to reach the solubility limit.

The results in Figure \ref{fig:SolSumFactor}, from the detailed parametric 
analysis in \cite{huff_key_2012}, it is clear that for 
solubility constants lower than the saturation threshold, the transport regime is solubility 
limited and the relationship between peak annual dose and solubility limit is 
strong.  Above the threshold, the transport regime is inventory limited 
instead.

In the corresponding parametric analysis of \Cyder performance, it was shown that the 
solubility sensitivity behavior closely matched that of the \gls{GDSM} 
sensitivity behaviors. Specifically, in Figure \ref{fig:sol_result}, a sharp turnover 
is seen where the solubility limit exceeds the point at which it limits 
movement. For increased solubility limits, release remains constant.

\begin{figure}[ht]
\begin{center}
\includegraphics[width=0.7\linewidth]{./Solubility_Summary_SolFactor.eps}
\caption[Solubility factor sensitivity in GDSM Clay model]{Solubility factor sensitivity. The peak annual dose due to an inventory, $N$, of each isotope. This result was acheived with a parametric analysis using a detailed model of a generic clay repository \ref{huff_key_2012}}
\label{fig:SolSumFactor}
\end{center}
\end{figure}

%\begin{figure}[ht]
%\begin{center}
%\includegraphics[width=0.7\linewidth]{./Solubility_Summary_Sol.eps}
%\caption[Solubility limit sensitivity in GDSM Clay model]{Solubility limit sensitivity. The peak annual dose due to an inventory, 
%$N$, of each isotope.}
%\label{fig:SolSum}
%\end{center}
%\end{figure}

\begin{figure}[htbp!]
\begin{center}
\includegraphics[width=0.7\linewidth]{./sol.eps}
\caption[Solubility Sensitivity in the Mixed Cell Model]{Sensitivity demonstration of solubility limitation in \Cyder for an arbitrary isotope assigned a variable solubility limit.}
\label{fig:sol_result}
\end{center}
\end{figure}
