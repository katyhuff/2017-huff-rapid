

Nuclear fuel cycle and and nuclear waste disposal decisions are
technologically coupled by repository capacity.  For this reason, dynamic
integration of a generic disposal model and a fuel cycle systems analysis
framework is necessary to illuminate performance distinctions of candidate
repository host media, designs, and engineering components in the context of
fuel cycle options. However, current nuclear fuel cycle simulators lack dynamic
repository performance analysis due to the computational burden of
high-fidelity hydrolgic contaminant transport models.

To fill this capability gap, the \Cyder radionuclide contaminant transport
framework was developed to provide a modular simulation tool for dynamic repository performance
analysis.  Within this framework, medium-fidelity hydrologic radionuclide
transport models were developed to support assessment appropriate for fuel
cycle simulation. Rapidly modeling hundreds of waste packages in a geologic
environment is enabled within this framework by a suite of four closed form
models for advective, dispersive, coupled, and idealized contaminant transport.
A summary of the framework and an overview of the mathematical models is
presented. Additionally, parametric demonstrations simulations performed with
\Cyder are presented here and compared to parametric simulations with the
a more detailed computational model, the Clay \gls{GDSM} developed by the
DOE-NE Used Fuel Disposition Campaign.
