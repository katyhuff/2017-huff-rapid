Nuclear fuel cycle and and nuclear waste disposal decisions are
technologically coupled.  However, current nuclear fuel cycle simulators lack dynamic
repository performance analysis due to the computational burden of
high-fidelity hydrolgic contaminant transport models. The \Cyder disposal 
envirionment and repository module was developed to fill this gap. It implements medium-fidelity hydrologic 
radionuclide transport models to support assessment appropriate for fuel
cycle simulation in the \Cyclus fuel cycle simulator. 

Rapid modeling of hundreds of discrete waste packages in a geologic environment 
is enabled within this module by a suite of four closed form models for 
advective, dispersive, coupled, and idealized contaminant transport: a 
Degradation Rate model, a Mixed Cell model, a Lumped Parameter model, and a 1-D 
Permeable Porous Medium model.  A summary of the \Cyder module, its 
timestepping algorithm, and the mathematical models implemented within it are 
presented. Additionally, parametric demonstrations simulations performed with 
\Cyder are presented and shown to demonstrate functional agreement with parametric 
simulations conducted in a standalone hydrologic transport model, the Clay 
Generic Disposal System Model developed by the Used Fuel Disposition Campaign 
Department of Energy Office of Nuclear Energy. 

% Add methods used, results, conclusions.

%For this reason, dynamic integration of a generic disposal model with a fuel 
%cycle systems analysis framework is necessary to illuminate performance 
%distinctions of candidate repository host media, designs, and engineering 
%components in the context of fuel cycle options. 
