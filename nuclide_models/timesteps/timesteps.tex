\subsection{Time Stepping Algorithm}\label{sec:time stepping}

In \Cyder, radionuclide contaminants flow outward from the central component,
usually the waste form. An implicit time stepping method arrives at the updated
state of each component, radially outward, as a function of both the past state
and the current state of the system. Mass balance is conducted in each
component at each time step. These calculations proceed from the innermost
component (e.g. the waste form) to the outermost component (e.g. the far field), with mass transfer calculations conducted
at the boundaries. As mass flows from inner components to outer components, the
mass balances in both components are updated.  Thus, nuclide release
information passes radially outward from the waste stream sequentially through
each containment layer to the geosphere in a generic geometry of the form in
Figure \ref{fig:repo_layout}.
The default timestep in \Cyclus, and therefore in \Cyder, is one month.

At each component interface where mass transfer occurs and within each component
where mass balances take place, the flow model is solved with the most up to
date information available.  To illustrate the algorithm by which mass flow
calculations are conducted through the system of components at each time step,
the phases of a single time step for a simple pair of components will be
described.

The flow of the timestepping algorithm is seen in Figures \ref{fig:vols}, 
\ref{fig:vols_trans}, and \ref{fig:vols_update} and is detailed further in the 
following sections.  For the remaining discussion, the source of material, $i$, is the 
inner component (i.e. the waste form) and the next destination of the material, 
$j$, is the adjacent outer component (i.e. the waste package). This example 
will be carried through the explanation of all five phases of the timestepping 
algorithm. 



\begin{figure}[htbp!]
  \begin{center}
    \def\svgwidth{.4\textwidth}
    \input{./nuclide_models/timesteps/vols.eps_tex}
  \end{center}
  \caption{Two components share an interface at $r_j$ and contain mass and
  concentration profiles at the beginning of timestep $t_n$.}
  \label{fig:vols}
\end{figure}

\begin{figure}[htbp!]
  \begin{center}
    \def\svgwidth{.4\textwidth}
    \input{./nuclide_models/timesteps/vols_trans.eps_tex}
  \end{center}
  \caption{The mass balance model in component k calculates the appropriate
  mass transfer based on boundary information from component j.}
  \label{fig:vols_trans}
\end{figure}

\begin{figure}[htbp!]
  \begin{center}
    \def\svgwidth{.4\textwidth}
    \input{./nuclide_models/timesteps/vols_update.eps_tex}
  \end{center}
  \caption{Based on the mass transfer, both components update their mass and
  concentration profiles based on their mass balance model.}
  \label{fig:vols_update}
\end{figure}



\FloatBarrier


\subsubsection{Phase 1: Initial Conditions}

At the beginning of a timestep, the initial conditions in both the source and
the sink are equal to the final updated state of the previous time step.
On the first time step, initial mass inventories of each component in
the repository system must be defined. In the example case of the source and 
sink, this might be

\begin{align}
 m_i(t_0) = 10 \mbox{[kg]}\\
 m_j(t_0) = 0.
\end{align}

\subsubsection{Phase 2: Interior Mass Balance}

The mass distribution, $m_i(\vec{r})$, and concentration profile, 
$C_i(\vec{r})$, in the interior source volume $i$ at time $t_n$ are calculated 
based on the initial condition, any influxes, and the physics of its mass 
balance model.  The four different mathematical models available for this mass 
distribution and concentration profile calculation are discussed in Section 
\ref{sec:mass_balance}.

The resulting mass distribution and concentration profiles fully inform
the conditions on the boundary at $r_i$. This information is made available
to the external component, $j$ in order to support the next phase, the mass
transfer calculation. In the example case of the source and sink, this step 
defines the concentration profile across each volume,

\begin{align}
C_i(r) = f(r)
C_j(r) = 0\\
\intertext{where}
\int f(r)dV = 10.
\end{align}

\subsubsection{Phase 3: Mass Transfer Calculation}

The mass transfer, $m_{ij}(t_n)$ from the source volume $i$ to the sink volume 
$j$ is calculated next, based on the up to date conditions in volume $i$ (where 
$0\le r \le r_i$) determined in Phase 2 and the initial conditions in volume 
$j$ (where $r_i \le r \le r_j$). The mass transfer is calculated according to 
the mass transfer mode preference of the mass balance model of volume $j$, as 
discussed in Section \ref{sec:mass_transfer}.

In the example case of the source and sink, mass transfered from $i$ to $j$ is 
determined based on conditions in $i$ and the mass transfer mode between $i$ 
and $j$. Said another way, the mass transfer rate $m_{ij}$ between $i$ and $j$ 
is equivalent to the time derivative $\dot{m}$ of the mass released from $i$ and 
simultaneously entering $j$,

\begin{align}
m_{ij}= -\dot{m_i} = \dot{m_j}.
\end{align}

\subsubsection{Phase 4: Exterior Mass Balance}

When a mass flux $m_{ij}$ is determined between volumes $i$ and $j$, the
mass is added to the exterior sink volume $j$. Accordingly, necessary updates
are made to the mass balance and concentration profile as discussed in Section
\ref{sec:mass_balance}. In the case of the source and sink, the mass change in
phase 3 is added to the external component, $j$,

\begin{align}
        m_j(t_n) &= m_j(t_{n-1}) + \int_{t_{n-1}}^{t_n}m_{ij}dt\\
                 &= m_j(t_{n-1}) + m_{ij}(t_n).
\end{align}

\subsubsection{Phase 5: Interior Mass Balance Update}

When a mass flux $m_{ij}(t_n)$ is determined between volumes $i$ and $j$, and
the mass added to the exterior sink volume $j$ (as in phase 4) it is also
extracted from the interior source volume $i$.  When the material is extracted
from the interior source volume, the contained mass distribution and
concentration profile are updated to reflect this change,

\begin{align}
  m_{i}(t_n^*) &= m_i(t_n) - m_{ij}(t_n).
\end{align}

