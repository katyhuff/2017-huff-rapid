\subsection{Time Stepping Algorithm}\label{sec:time stepping}

In \Cyder, radionuclide contaminants flow outward from the central 
Component. Mass balance is conducted in each Component at each time step. These 
calculations proceed from the innermost Component to the outermost Component, 
with mass transfer calculations conducted at the boundaries. As mass flows from 
inner components to outer components, the mass balances in both components are 
updated.  Thus, nuclide release information passes radially outward from the 
waste stream sequentially through each containment layer to the geosphere.  This 
implicit time stepping method arrives at the updated state of each Component, 
radially outward, as a function of both the past state and the current state of 
the system.  

At each component interface where mass transfer occurs and within each component 
where mass balances take place, the flow model is solved with the most up to 
date information available.  To illustrate the algorithm by which mass flow 
calculations are conducted through the system of components at each time step, 
the phases of a single time step for a simple pair of components will be 
described. For the remaining discussion, the source, $i$, is the inner and the 
sink, $j$, is the outer component. 

\subsubsection{Phase 1: Initial Conditions}

The initial conditions in both the source and the sink at the beginning of a 
time step are equal to the final updated state of the previous time step. On the 
first time step, the global initial state of the repository system is used. 

\subsubsection{Phase 2: Interior Mass Balance}

The mass distribution, $m_i(\vec{r})$, and concentration profile, $C_i(\vec{r})$, in the interior source volume 
$i$ is solved based on the initial condition, any influxes, and the physics of 
its mass balance model.  This calculation results in a contaminant mass 
distribution and concentration profile within the volume $i$ at time $t_n$.  
For each of the models, the calculation behind this mass distribution and 
concentration profile is discussed in Section \ref{sec:mass_balance}.

This mass distribution and concentration profile fully inform 
the conditions on the boundary at $r_i$ and this information is made available 
to the external component, $j$.


\subsubsection{Phase 3: Mass Transfer Calculation}

The mass transfer from the source volume $i$ to the sink volume $j$ is 
calculated next, based on the up to date conditions in volume $i$ (where $0\le r \le r_i$) 
determined in Phase 2 and the initial conditions in volume $j$ (where $r_i \le 
r \le r_j$). The mass transfer is calculated according to the mass transfer mode 
preference of the mass balance model of volume $j$.  

%The Degradation Rate and Mixed Cell
Two of the mass balance models (Degradation Rate and Mixed Cell) can be 
parameterized to utilize an explicit mass transfer mode that captures either 
advection, dispersion, or coupled flow.  The other two (Lumped Parameter and 
One Dimensional PPM) models use an implicit method by which the incoming mass 
flux is determined based on the expected concentration profile resulting from 
the internal Dirichlet boundary condition at $r_i$. 

\subsubsection{Phase 4: Exterior Mass Balance}

When a mass flux $m_{ij}(t_n)$ is determined between volumes $i$ and $j$, the 
mass is added to the exterior sink volume $j$. Accordingly, necessary updates 
are made to the mass balance and concentration profile as discussed in Section 
\ref{sec:mass_balance}.

\subsubsection{Phase 5: Interior Mass Balance Update}

When a mass flux $m_{ij}(t_n)$ is determined between volumes $i$ and $j$, and 
the mass added to the exterior sink volume $j$ (as in phase 4) it is also 
extracted from the interior source volume $i$.  When the material is extracted 
from the interior source volume, the contained mass distribution and 
concentration profile are updated to reflect this change,

\begin{align}
  m_{i}^*(t_n) &= m_i(t_n) - m_{ij}(t_n).
\end{align}

%\subsubsection{old text}
%
%That is, in Component $j$, some Component in a nested series, the mass flux 
%entering the Component at time $t_n$ is found from the initial state of the cell 
%at time $t_n$, the inner boundary 
%condition at time $t_n$ and the outer boundary condition at $t_{n-1}$.  
%
%\begin{align}
%  \dot{m}_{ij}^n &= f( m_j(t_{n-1}) , BC_i(t_n) , BC_j(t_{n-1}) . . . ) \nonumber\\
%  \intertext{where}
%  m_{ij}(t_n) &= \mbox{ contaminant mass flux from component i to j }[kg/time step]\nonumber\\
%  BC_i(t_n)  &= \mbox{ inner conditions at }r_i\mbox{, and time }t_n \nonumber \\
%  BC_j(t_{n-1})  &= \mbox{ outer conditions at }r_j\mbox{, and time }t_{n-1} \nonumber\\
%  f &= \mbox{ functional form of contaminant transport into j. }\nonumber
%\end{align}
%
%Once the mass flux into the component is found, the mass is removed from the 
%inner cell, updating its state in preparation for the next time step.
%
%\begin{align}
%  m_i^\dagger(t_n)  &= m_i(t_n)  - m_{ij}(t_n) 
%  \intertext{where}
%  m_i^\dagger(t_n)  &= \mbox{ updated mass in component i }[kg]
%\end{align}
%
%In this way, the contained mass in the component is described as
%\begin{align}
%  m_j(t_n)  &= m_j(t_{n-1})  + \dot{m}_j(t_n) . \nonumber
%\end{align}
%
%Resulting concentration profiles across the component can then be calculated 
%and one can solve, numerically, for the outer boundary condition at $t_n$ 
%
%\begin{align}
%  BC_j(t_n) &= g\left( m_j(t_n) , C_j(t_n) \right)\nonumber\\
%  g &= \mbox{functional form of contaminant transport across j}\nonumber
%\end{align}
%
%This boundary condition can, in turn, be used by the component external to it, $k$ as the $t_n$ 
%inner boundary condition of its own solution and so on.
%
