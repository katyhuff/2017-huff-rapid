\Cyder conducts radionuclide contaminant transport through a generic geologic 
repository concept to determine the 
contaminants expected to reach the environment. This key calculation 
informs repository performance assesment metrics related to containment and 
environmental impact.

To acheive this, \Cyder models engineered and natural containment barriers as 
distinct control volumes. These \emph{components} are arranged in a regular 
grid at a single vertical depth within a geologic component.  Component mass 
inventory is a simple sum of in and out flows while mass distribution within 
the component is determined by the dominant physics of the mass balance model 
selected for that volume.  Adjacent components share mass transfer interfaces 
across which mass transfer is calculated based on internal component mass 
inventory and distribution. 

In \Cyder, the mass transfer and mass balance solution follows an implicit 
\emph{time stepping algorithm}. The solution behavior is determined by selecting 
among \emph{mass balance models} within the components and selecting 
among \emph{mass transfer modes} at boundaries between them. This section will describe 
the mathematics behind these three aspects of the \Cyder paradigm, beginning 
with the phases of the time stepping algorithm.
