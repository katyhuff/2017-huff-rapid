\subsection{Implications For Waste Form Modeling}

Though the Waste Form Component can be modeled by any of the available 
NuclideModels, the Degradation Rate based or Mixed Cell radionuclide transport 
models are preferred for modeling of the Waste Form Component.  This is 
because, in repository performance assessments, waste form dissolution is 
typically modeled as instantaneous or rate based. Dissolution related release 
is historically modeled as congruent, solubility limited, or both, with some 
radionuclides becoming immediately accessible, and some tending to remain in 
the fuel matrix. 

\subsection{Implications For Waste Package Modeling}

Though the Waste Package Component can be modeled by any of the available 
NuclideModels, the simple Degradation Rate based model is strongly preferred.
Waste package time to failure is dependent on water contact and heat, 
but is historically modeled probabilistically, or at a constant rate.
Accordingly, waste package degradation in repository performance is either 
neglected entirely, instantaneous and complete (a delay before full release), 
or partial and constant (a constantly present hole in the package). 


\subsection{Implications For Buffer Modeling}

Diffusion is the primary mechanism for nuclide transport through the 
buffer Component of the repository system. While the buffer may degrade, the 
near field has historically been modeled in as much hydrologic detail as 
possible. For this reason, the Lumped Parameter or One Dimensional Permeable 
Porous Medium nuclide transport models are preferred.

\subsection{Implications for Geologic Environment Modeling}

In most of the saturated, low permeability environments being considered, 
diffusion is the primary mechanism for nuclide transport through the geologic 
medium Component of the repository system. While the near field may degrade, 
the far field should be modeled in detail if possible. For this reason, the 
One Dimensional Permeable Porous Medium nuclide transport 
model is preferred.
