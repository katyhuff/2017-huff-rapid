The mass transfer interfaces between the mass balance models are essential to 
the understanding of the \Cyder paradigm.  Depending on the mass balance model 
selected in the external of two components, mass transfer into that component is 
either explicit or implicit.  

In the explicit mode, the mass transfer mode is chosen by the user among advective, dispersive, 
coupled or fixed flux and is calculated based on the conditions at the transfer 
boundary. The inventory in the components is then updated based on this transfer 
rate. While all components enable this on their outer boundary, only the mass 
balance models that are 0-dimensional in space (the Degradation Rate model and 
the Mixed Cell model) require explicit transfer on their inner boundary.

In the implicit mode, the mass balance model of the external component determines 
the inventory based on boundary conditions provided by the internal component. 
The appropriate mass is then transferred to accomplish the change in inventory.

In groundwater transport, contaminants are transported by dispersion and 
advection. It is customary to define the combination of molecular diffusion and 
mechanical mixing as the dispersion tensor, $D$, such that, for a conservative 
solute (infinitely soluble and non-sorbing), the mass conservation equation 
becomes \cite{schwartz_fundamentals_2004, wang_introduction_1982, 
van_genuchten_analytical_1982}:

     \begin{align}
      J &= J_{dis} + J_{adv}\nonumber
      \intertext{where}
      J_{dis} &= \mbox{ Total Dispersive Mass Flux }[kg/m^2/s]\nonumber\\
      &= -\theta(D_{mdis} + \tau D_m)\nabla C \nonumber\\ 
      &= -\theta D\nabla C \nonumber\\
      J_{adv} &= \mbox{ Advective Mass Flux }[kg/m^2/s]\nonumber\\
      &= \theta vC\nonumber\\
      \tau &= \mbox{ Tortuosity }[-] \nonumber\\
      \theta &= \mbox{ Porosity }[-] \nonumber\\
      D_m &= \mbox{ Molecular diffusion coefficient }[m^2/s]\nonumber\\
      D_{mdis} &= \mbox{ Coefficient of mechanical dispersivity}[m^2/s]\nonumber\\
      D &= \mbox{ Effective Dispersion Coefficient }[m^2/s]\nonumber\\
      C &= \mbox{ Concentration }[kg/m^3]\nonumber\\
      v &= \mbox{ Fluid Velocity in the medium }[m/s].\nonumber
    \end{align}

For uniform flow in $\hat{k}$, 
    \begin{align}
      J &=\left(-\theta D_{xx} \frac{\partial C}{\partial x}
             \right)\hat{\imath}
             + \left( -\theta D_{yy} \frac{\partial C}{\partial y}
            \right)\hat{\jmath}\nonumber\\
            &+ \left( -\theta D_{zz} \frac{\partial C}{\partial z}
             + \theta v_zC \right)\hat{k}.
      \label{unidirflow}
    \end{align}

Solutions to this equation can be categorized by their boundary conditions.  
Those boundary conditions serve as the interfaces between components in the 
\Cyder library of nuclide transport models by way of advective, dispersive, 
coupled, and fixed fluxes.  This is supported by implementation in which 
vertical advective velocity, $v_z$, is uniform throughout the system and in which 
parameters such as the dispersion coefficient are known for each component. 




