\subsubsection{Explicit Dispersion Dominated Mass Transfer}\label{sec:diff_mass_transfer}


The second type, specified dispersive flux, or Neumann type boundary conditions describe a full set of 
concentration gradients at the boundary of the domain,
    \begin{align}
      \frac{\partial C(\vec{r},t)}{\partial r}\Big|_{\vec{r}\in\Gamma} &= f(t)\\
      f(t) &= \mbox{ known function }.\nonumber
    \end{align}

The Neumann boundary condition can be provided at the external boundary of any 
mass balance model,
\begin{align}
\frac{\partial C}{\partial z}\Bigg|_{z=r_j} &= \mbox{ fixed concentration gradient in j at }r_j\mbox{ and } t_n [kg/m^3/s].\nonumber
\end{align}


For mass balance models that are 0-dimensional in space (i.e. the Degradation 
Rate model and the Mixed Cell model), which lack spatial variation in the 
concetration profile, the differential must be approximated. Taking the 
center-to-center difference between adjacent components is one convenient way 
to make this approximation, and is the method implemented in \Cyder, such that 

\begin{align}
\frac{\partial C(z,t_n)}{\partial z}\Bigg|_{z=r_j} &= \frac{C_k(r_{k-1/2},t_{n-1}) - C_j(r_{j-1/2}, t_n)}{r_{k-1/2} - r_{j-1/2}}
\intertext{where}
r_{j-1/2} &= r_{j} - \frac{r_{j} - r_i}{2}\nonumber\\
r_{k-1/2} &= r_{k} - \frac{r_{k} - r_j}{2}.\nonumber
\end{align}

However, for mass balance models that are 1-dimensional in space (i.e. the 
Lumped Parameter model and the One Dimensional PPM model), the derivative is 
taken based on the concentration profile in the internal component as it 
approaches the boundary.  In component $j$, if it is a lumped parameter model, 
the profile is assumed to be a linear relationship between $C_{in}$ and 
$C_{out}$, the gradient is

\begin{align} 
\frac{\partial C(z,t_n)}{\partial z}\Bigg|_{r_i\le z\le j} &= \frac{C_{out} - C_{in}}{r_{j} - r_{i}}.
\end{align}

For the one dimensional permeable porous medium model, the analytical 
derivative of equation \eqref{simple_genuchten} is evaluated at $r_j$.

%\begin{align}
%derivative..
%\end{align}

For mass transfer into the Degradation Rate and Mixed Cell models, the Neumann 
boundary condition can be chosen to enforce a dispersive flux on the inner 
boundary. This choice is appropriate when the user expects a primarily 
dispersive flow across the boundary. The dispersive flux in one dimension, 
\begin{align}
      J_{dis} &= \mbox{ Total Dispersive Mass Flux }[kg/m^2/s]\nonumber\\
      &= -\theta D\frac{\partial C}{\partial z} \nonumber
\end{align}
relies on the fixed gradient Neumann boundary condition at the interface. 
The resulting mass transfer into the Degradation Rate or Mixed Cell model is, 
therefore, 

\begin{align}
m_{jk}(t_n) &= - A\Delta t \theta_k D \frac{\partial C(z,t_n)}{\partial z}|_{z=r_j}.
\end{align}
