\subsubsection{Lumped Parameter Radionuclide Mass Balance Model}\label{sec:lumped}

For systems in which the flow is sufficiently slow to be assumed constant over 
a time step, it is possible to model a system of volumes as a connected lumped 
parameter models (Figure \ref{fig:lumpedseries}). 
The Lumped Parameter mass balance model implements a response function model 
based on this lumped parameter interpretation and capable of Piston Flow, 
Exponential, and Dispersion response functions from Maloszewski and Zuber 
\cite{maloszewski_lumped_1996}.

\begin{figure}[htbp!]
  \begin{center}
    \def\svgwidth{\columnwidth}
    \input{./nuclide_models/mass_balance/lumped/lumpedseries.eps_tex}
  \end{center}
  \caption{A system of volumes can be modeled as lumped parameter models in 
  series.}
  \label{fig:lumpedseries}
\end{figure}

Each lumped parameter component is modeled according to a 
relationship between the incoming concentration, $C_{in}(t)$, and the outgoing 
concentration, $C_{out}(t)$, 

\begin{align}
  C_{out}(t) &= \int_0^\infty C_{in}(t-t')g(t')e^{-\lambda t'}dt'
  \label{lumped2}
  \intertext{where}
  t'  &= \mbox{ transit time }[s]\nonumber\\
  g(t')  &= \mbox{ response function, a.k.a. transit time distribution}[-]\nonumber\\
  \lambda &= \mbox{ radioactive decay constant }[s^{-1}].\nonumber
\end{align}

Selection of the response function is usually based on experimental tracer 
results in the medium at hand. If such detailed transport  data is not available, 
functions used commonly in chemical engineering applications 
\cite{maloszewski_lumped_1996} include the Piston Flow Model (PFM), which 
approximates pure advection,

\begin{align}
  g_{PFM}(t') &= \delta{(t'-t_t))},
\end{align}
the Exponential Model (EM) which approximates a well-mixed flow case,
\begin{align}
  g_{EM}(t') &= \frac{1}{t_t}e^{-\frac{t'}{t_t}}
\end{align}
and the so-called Dispersion Model (DM), which actually approximates the solution to both 
advective and dispersive transport,
\begin{align}
  g_{DM}(t') &= \left( \frac{\emph{Pe }t_t}{4\pi t'} \right)^{\frac{1}{2}}
  \frac{1}{t'}e^{- \frac{\emph{Pe }t_t\left( 1- \frac{t'}{t_t}  \right)^2} 
  {4t'}}, \intertext{where}
  \emph{Pe}  &= \mbox{ Peclet number for mass diffusion }[-]\nonumber\\
  t_t  &= \mbox{ mean tracer age }[s]\nonumber\\
    &= t_w \mbox{ if there are no stagnant areas }\nonumber\\
  t_w  &= \mbox{ mean residence time of water }[s]\nonumber\\
       &= \frac{V_m}{Q}\nonumber\\
       &= \frac{z}{v_z}\nonumber\\
       &= \frac{z\theta_e}{q}\nonumber
  \intertext{in which}
  V_m  &= \mbox{ mobile water volume }[m^3]\nonumber\\
  Q    &= \mbox{ volumetric flow rate }[m^3/s]\nonumber\\
  z    &= \mbox{ average travel distance in flow direction }[m]\nonumber\\
  v_z  &= \mbox{ mean water velocity}[m/s]\nonumber\\
  q    &= \mbox{ Darcy Flux }[m/s]\nonumber\\
  \theta_e &= \mbox{ effective (connected) porosity }[\%].\nonumber
\end{align}

The latter of these, the Dispersion Model satisfies the one dimensional 
advection-dispersion equation, and is therefore the most physically relevant 
for this application. A constant inlet concentration is assumed over the span 
of a time step such that $C_{in}(t) = C_0$, and the solutions to these for 
constant concentration at the source boundary are given in Maloszewski and 
Zuber \cite{maloszewski_lumped_1996} accordingly, 

\begin{align}
  C_{out}(t) &=\begin{cases}
    PFM & C_0e^{-\lambda t_t}\\
    EM  & \frac{C_0}{1+\lambda t_t}\\
    DM & C_0e^{\frac{\emph{Pe}}{2}\left(1-\sqrt{1+\frac{4\lambda 
    t_t}{\emph{Pe}}}\right)}.
  \end{cases}
  \label{lumpedsolns}
\end{align}

Since \Cyclus handles decay outside of \Cyder, the use of these models relies on a 
reference transit time and decay constant supplied by the user. The behavior of 
the reference isotope, in this way, fully defines the behavior of all isotopes.

It is important to note that a linear concentration profile is assumed between 
the inlet and the outlet of a given Component in \Cyder,

\begin{align}
  C(z,t) = C_{in}(t)  + \frac{C_{out}(t) - C_{in}(t)}{z_{out} - z_{in}}(z-z_{in}).
\end{align}
This is an approximation that could be improved by direct use of the response 
functions themselves, under a change of variables from time to length. 

