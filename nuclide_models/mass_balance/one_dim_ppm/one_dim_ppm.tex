\subsubsection{One Dimensional Permeable Porous Medium Radionuclide Mass Balance 
Model}\label{sec:one_dim_ppm}

Various solutions to the advection dispersion equation  
\eqref{unidirflow} have been published for both the first and third types of 
boundary conditions. The third, Cauchy type, is more mass conservative, and is 
the primary kind of boundary condition used at the source for the model 
implementation in \Cyder.
Abstraction results informed modifications to the implementation of an 
analytic solution to the one dimensional advection-dispersion equation with 
a finite domain and Cauchy and Neumann boundary conditions at the inner and outer 
boundaries, respectively. 

The conceptual model in Figure \ref{fig:1dinf} represents solute transport in 
one dimension with unidirectional flow upward (a conservative assumption) and a 
finite boundary condition in the positive flow direction. 
In \Cyclus, radioactive decay is handled external to the components, so there is 
no need to include production or decay.  An approximate solution for these conditions  
made by Brenner \cite{brenner_diffusion_1962} is described below as 
it is given in van Genuchten et. al, \cite{van_genuchten_analytical_1982}, 

\begin{figure}[h!]
  \begin{center}
    \def\svgwidth{0.7\columnwidth}
    \input{./nuclide_models/mass_balance/one_dim_ppm/1dfin.eps_tex}
  \end{center}
  \caption[1D finite advection dispersion solution.]{A one dimensional, 
  finite, unidirectional flow solution with Cauchy and Neumann boundary 
conditions}
  \label{fig:1dinf}
\end{figure}

For the boundary conditions, 
\begin{align}
  -D \frac{\partial C}{\partial z}\big|_{z=0} + v_zc &= \begin{cases}
    v_zC_0  &  \left( 0<t<t_0 \right)\\
    0  &  \left( t>t_0 \right)\\
  \end{cases}
\intertext{and}
  \frac{\partial C}{\partial z}\big|_{z=L} &= 0
  \intertext{and the initial condition,}
  C(z,0) &= C_i,
  \label{1dinfBC}
\end{align}
\begin{align}
  \intertext{the solution is given as }
  C(z,t) &= \begin{cases} 
  C_i + \left(C_0 - C_i\right)A\left(z,t\right) & 0<t\le t_0\\
  C_i + \left(C_0 - C_i\right)A\left(z,t\right) - C_0A(z,t-t_0) & t\ge t_0.
  \end{cases}
\end{align}

For the vertical flow coordinate system, $A$ is defined as
\begin{align}
A(z,t) =& \left(\frac{1}{2}\right)\erfc{\left[\frac{Rz-vt}{2\sqrt{DRt}}\right]} \nonumber\\
&+ \left(\frac{v^2t}{\pi RD}\right)^{1/2}\exp{\left[-\frac{(Rz-vt)^2}{4DRt}\right]}\nonumber\\ 
&- \frac{1}{2} \left(1+\frac{vz}{D} + \frac{v^2t}{DR}\right) \exp{\left[\frac{vz}{D}\right]}\erfc{\left[\frac{Rz+vt}{2\sqrt{DRt}}\right]}\nonumber\\
&+ \left(\frac{4v^2t}{\pi RD}\right)^{1/2}\left[1+\frac{v}{4D}\left(2L-z+\frac{vt}{R}\right)\right]\exp{\left[\frac{vL}{D} - \frac{R}{4Dt}\left(2L-z+\frac{vt}{R}\right)^2\right]}\nonumber\\
&- \frac{v}{D}\left[2L - z + \frac{3vt}{2R} + \frac{v}{4D}\left(2L - z + \frac{vt}{R}\right)^2\right]\exp{\left[\frac{vL}{D}\right]}\erfc{\left[\frac{R(2L-z) + vt}{2\sqrt{DRt}}\right]}
\label{simple_genuchten}
\intertext{where}
L =& \mbox{Extent of the solution domain }[m]\nonumber\\
R =& \mbox{Retardation factor }[-].\nonumber
\end{align}


