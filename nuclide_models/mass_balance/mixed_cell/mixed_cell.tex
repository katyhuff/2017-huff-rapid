\subsubsection{Mixed Cell Radionuclide Mass Balance Model}\label{sec:mixed_cell}

Slightly more complex, the Mixed Cell model incorporates the influence of
porosity, elemental solubility limits, and sorption in addition to the
degradation behavior of the Degradation Rate model. A graphical representation
of the discrete sub-volumes in the mixed cell model is given in Figure
\ref{fig:deg_sorb_volumes}.

\begin{figure}[h!]
  \begin{center}
    \def\svgwidth{\columnwidth}
    \begin{figure}[h!]
  \begin{center}
    \def\svgwidth{\columnwidth}
    \begin{figure}[h!]
  \begin{center}
    \def\svgwidth{\columnwidth}
    \input{./nuclide_models/mass_balance/mixed_cell/deg_sorb_volumes.eps_tex}
  \end{center}
  \caption[Constituents of a Mixed Cell Control Volume]{The degraded volume is
  modeled as a degraded solid volume, $V_{ds}$, and a degraded fluid volume,
  $V_{df}$. The intact volume is modeled as an intact solid volume, $V_{is}$, and
  an intact fluid volume $V_{if}$.  Only contaminants in $V_{df}$ are available
  for transport.}
  \label{fig:deg_sorb_volumes}
\end{figure}


  \end{center}
  \caption[Constituents of a Mixed Cell Control Volume]{The degraded volume is
  modeled as a degraded solid volume, $V_{ds}$, and a degraded fluid volume,
  $V_{df}$. The intact volume is modeled as an intact solid volume, $V_{is}$, and
  an intact fluid volume $V_{if}$.  Only contaminants in $V_{df}$ are available
  for transport.}
  \label{fig:deg_sorb_volumes}
\end{figure}


  \end{center}
  \caption[Constituents of a Mixed Cell Control Volume]{The degraded volume is
  modeled as a degraded solid volume, $V_{ds}$, and a degraded fluid volume,
  $V_{df}$. The intact volume is modeled as an intact solid volume, $V_{is}$, and
  an intact fluid volume $V_{if}$.  Only contaminants in $V_{df}$ are available
  for transport.}
  \label{fig:deg_sorb_volumes}
\end{figure}



After some time degrading, the total volume in the degraded region can be 
expressed as in equation \eqref{deg_volumes}. Additionally, given a volumetric 
porosity, $\theta$, the intact and degraded volumes can also be described in 
terms of their constituent solid matrix and pore fluid volumes,

\begin{align}
V_d(t_n) &= \mbox{ degraded volume at time }t_n [m^3]\nonumber\\
          &= V_{df}(t_n) + V_{ds}(t_n)
\intertext{where}
V_{df}(t_n) &= \mbox{ degraded fluid volume at time }t_n[m^3]\nonumber\\
       &= \theta V_d(t_n)\\
       &= \theta d(t_n) V_T
\end{align}
\begin{align}
V_{ds}(t_n) &= \mbox{ degraded solid volume at time }t_n [m^3]\nonumber\\
       &= (1-\theta) V_d(t_n)\\
       &= (1-\theta) d(t_n) V_T
\end{align}
\begin{align}
V_i(t_n) &= \mbox{ intact volume at time }t_n [m^3]\nonumber\\
       &= V_{if}(t_n) + V_{is}(t_n)
\end{align}
\begin{align}
V_{if}(t_n) &= \mbox{ intact fluid volume at time }t_n [m^3]\nonumber\\
       &= \theta V_i(t_n)\\
       &= \theta (1-d(t_n))V_T
\intertext{and}
V_{is}(t_n) &= \mbox{ intact solid volume at time }t_n [m^3]\nonumber\\
       &= (1-\theta) V_i(t_n)\\
       &= (1-\theta) (1-d(t_n))V_T.
\end{align}

This model distributes contaminant masses throughout each sub-volume of the
component. Contaminant
masses and concentrations can therefore be expressed with notation indicating
in which volume they reside, such that

\begin{align}
C_{df} &= \frac{m_{df}}{V_{df}} \label{c_df}\\
C_{ds} &= \frac{m_{ds}}{V_{ds}} \label{c_ds}\\
C_{if} &= \frac{m_{if}}{V_{if}} \label{c_if}\\
C_{is} &= \frac{m_{is}}{V_{is}}.  \label{c_is}
\end{align}

The contaminant mass in the degraded fluid is the contaminant mass that is
treated as ``available'' to adjacent components.

\paragraph{Sorption}

The mass in all volumes exists in both sorbed and non-sorbed phases. The
relationship between the sorbed mass concentration in the solid phase (e.g. the
pore walls),

\begin{align}
s &=\frac{\mbox{ mass of sorbed contaminant} }{ \mbox{mass of total solid phase }}
\label{solid_conc}
\end{align}
and the dissolved liquid concentration,
\begin{align}
C &=\frac{\mbox{ mass of dissolved contaminant} }{ \mbox{volume of total liquid phase }}
\label{liquid_conc}
\end{align}
can be characterized by a sorption ``isotherm'' model. A sorption isotherm
describes the equilibrium relationship between the amount of material bound to
surfaces and the amount of material in the solution. The Mixed Cell mass
balance model uses a linear isotherm model.

With the linear isotherm model, the mass of contaminant sorbed into the
solid phase can be found
\cite{schwartz_fundamentals_2004}, according to the relationship
\begin{align}
s_p &= K_{dp} c_{p}
\label{linear_iso}
\intertext{where}
s_p &= \mbox{ the solid concentration of isotope p }[kg/kg]\nonumber\\
K_{dp} &= \mbox{ the distribution coefficient of isotope p}[m^3/kg]\nonumber\\
C_p &= \mbox{ the liquid concentration of isotope p }[kg/m^3].\nonumber
\end{align}

Thus, from \eqref{solid_conc},

\begin{align}
s_{dsp} &= K_{dp} C_{dfp}\nonumber\\
         &= \frac{K_{dp}m_{dfp}}{V_{df}}\nonumber
\intertext{where}
s_{dsp} &= \mbox{ isotope p concentration in degraded solids } [kg/kg] \nonumber\\
C_{dfp} &= \mbox{ isotope p concentration in degraded fluids } [kg/m^3]. \nonumber
\end{align}

In this model, sorption is taken into account throughout the volume. In the
intact matrix, the contaminant mass is distributed between the pore walls and
the pore fluid by sorption.  So too, contaminant mass released from the intact
matrix by degradation is distributed between dissolved mass in the free fluid
and sorbed mass in the degraded and precipitated solids.

To begin solving for the boundary conditions in this model, the amount of non-sorbed
contaminant mass in the degraded fluid volume must be found. Dropping the
isotope subscripts and beginning with equations \eqref{c_df} and \eqref{linear_iso},

\begin{align}
m_{df} &= C_{df}V_{df}
\intertext{and assuming the sorbate is in the degraded solids}
m_{df} &= \frac{s_{ds}V_{df}}{K_d},\nonumber\\
\intertext{then applying the definition of $s_{ds}$ and $m_{ds}$}
m_{df} &= \frac{\frac{m_{ds}}{m_T}V_{df}}{K_d}\nonumber\\
       &= \frac{(dm_T-m_{df})V_{df}}{K_dm_T}.\nonumber
\intertext{This can be rearranged to give}
m_{df} &= \frac{dV_{df}}{K_d}\frac{1}{\left( 1+ \frac{V_{df}}{K_dm_T}\right)}\nonumber\\
       &= \frac{dV_{df}}{\left( K_d+ \frac{V_{df}}{m_T}\right)}.
\intertext{Finally, in terms of total volume,}
m_{df} &= \frac{d^2 \theta V_T}{dK_d + \frac{d \theta V_T}{m_T}}.
       \label{sorption}
\end{align}

\paragraph{Solubility}
Dissolution of the contaminant into the
available fluid volume is constrained by the
elemental solubility limit.
The reduced mobility of radionuclides with lower
solubilities can be modeled \cite{hedin_integrated_2002} as a reduction in the
amount of solute available for transport, thus:

\begin{align}
  m_{i}(t)&\le V(t)C_{sol, i}\label{sol_lim}
  \intertext{where}
  m_{i} &= \mbox{ mass of isotope i in volume }V [kg]\nonumber\\
  V &= \mbox{ a distinct volume of fluid } [m^3]\nonumber\\
  C_{sol, i} &= \mbox{ the maximum concentration of i } [kg\cdot m^{-3}].\nonumber
\end{align}

That is, the mass $m_{1i}$ in  kg of a radionuclide $i$ dissolved into the waste package
void volume $V_1$ in m$^3$, at a time t, is limited by the solubility limit,
the maximum concentration, $C_{sol}$ in kg/m$^3$ at which that radionuclide is
soluble \cite{hedin_integrated_2002}.


The final available mass is therefore the $m_{df}$ from equation
\eqref{sorption} constrained by:
    \begin{align}
      m_{df,i} &\leq V_{df} C_{sol,i}
      \label{solubility}
    \intertext{where}
      m_{df,i} &= \mbox{ solubility limited mass of isotope i in volume }V_{df} [kg]\nonumber\\
      C_{sol,i} &= \mbox{ the maximum dissolved concentration limit of i }[kg/m^3].\nonumber
    \end{align}


