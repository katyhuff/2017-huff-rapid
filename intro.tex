\section{Introduction}\label{sec:introduction}
Repository performance metrics provide an important basis for comparison among
p tential nuclear fuel cycles.
For this reason, dynamic integration of a generic disposal model with a fuel 
cycle systems analysis framework is necessary to illuminate performance 
distinctions of candidate repository host media, designs, and engineering 
components in the context of fuel cycle options. 
However, the computational
burden of robust repository performance analysis has previously not been
compatible with fuel cycle simulation.
Therefore, current nuclear fuel cycle simulators
lack coupled repository performance analysis capabilities.

Most current tools treat the waste disposal
phase of fuel cycle analysis statically in post processing by reporting
values such as mass, volumes, radiotoxicity, or heat production of accumulated
\gls{SNF} and \gls{HLW}. Such tools
(e.g.,
\gls{NUWASTE} \cite{abkowitz_nuclear_2011},
\gls{DANESS} \cite{van_den_durpel_daness:_2006},
\gls{NFCSim} \cite{schneider_nfcsim_2004}, and
ORION \cite{gregg_orion_2011})
fail to address the dynamic impact of those waste streams on the performance of the
geologic disposal system \cite{wilson_comparing_2011}.  Two tools, \gls{COSI}
\cite{boucher_international_2010} and \gls{VISION} \cite{yacout_vision_2006,
wilson_comparing_2011, radel_repository_2007, boucher_international_2010},
dynamically perform heat based capacity calculations.
However, those calculations are applicable only for specific
repository concepts and cannot inform sensitivity to alternate geologic disposal
system characteristics.

The \Cyder software library \cite{huff_cyder_2013} and its radionuclide
contaminant transport models were  developed to fill this capability gap.  To
enable dynamic analysis of waste metrics, \Cyder provides medium fidelity
models to conduct repository performance analysis on efficient timescales
appropriate for fuel cycle analyses. It has been implemented as a Facility
compatible with version 0.3 of the \Cyclus framework
\cite{wilson_cyclus:_2012}, but can also be used as a standalone library. An
overview of the \Cyder framework and mathematical descriptions of its
radionuclide transport models appear in Section \ref{sec:nuclide_models}.

Parametric demonstration simulations performed with \Cyder were also conducted
within \Cyclus for verification purposes. Those results are presented in
Section \ref{sec:results} alongside comparable parametric simulations conducted
using a more detailed computational model, the Clay \gls{GDSM}, which was
developed by the \gls{UFD} Campaign within the \gls{DOE} Office of Nuclear
Energy \cite{clayton_generic_2011} and relies on the GoldSim simulation
environment \cite{golder_associates_goldsim_2010-1} and its contaminant
transport module \cite{golder_associates_goldsim_2010-1}.

