%        File: nuc_models.tex
%     Created: Wed Oct 09 02:00 PM 2013 P
% Last Change: Wed Oct 09 02:00 PM 2013 P
%

%
% Copyright 2007, 2008, 2009 Elsevier Ltd
%
% This file is part of the 'Elsarticle Bundle'.
% ---------------------------------------------
%
% It may be distributed under the conditions of the LaTeX Project Public
% License, either version 1.2 of this license or (at your option) any
% later version.  The latest version of this license is in
%    http://www.latex-project.org/lppl.txt
% and version 1.2 or later is part of all distributions of LaTeX
% version 1999/12/01 or later.
%
% The list of all files belonging to the 'Elsarticle Bundle' is
% given in the file `manifest.txt'.
%

% Template article for Elsevier's document class `elsarticle'
% with numbered style bibliographic references
% SP 2008/03/01
%
%
%
% $Id: elsarticle-template-num.tex 4 2009-10-24 08:22:58Z rishi $
%
%
\documentclass[preprint,12pt]{elsarticle}

% Use the option review to obtain double line spacing
% \documentclass[preprint,review,12pt]{elsarticle}

% Use the options 1p,twocolumn; 3p; 3p,twocolumn; 5p; or 5p,twocolumn
% for a journal layout:
% \documentclass[final,1p,times]{elsarticle}
% \documentclass[final,1p,times,twocolumn]{elsarticle}
% \documentclass[final,3p,times]{elsarticle}
% \documentclass[final,3p,times,twocolumn]{elsarticle}
% \documentclass[final,5p,times]{elsarticle}
% \documentclass[final,5p,times,twocolumn]{elsarticle}

% if you use PostScript figures in your article
% use the graphics package for simple commands
% \usepackage{graphics}
% or use the graphicx package for more complicated commands
\usepackage{graphicx}
% or use the epsfig package if you prefer to use the old commands
% \usepackage{epsfig}

% The amssymb package provides various useful mathematical symbols
\usepackage{amssymb}
% The amsthm package provides extended theorem environments
% \usepackage{amsthm}
\usepackage{amsmath}

% The lineno packages adds line numbers. Start line numbering with
% \begin{linenumbers}, end it with \end{linenumbers}. Or switch it on
% for the whole article with \linenumbers after \end{frontmatter}.
\usepackage{lineno}

% I like to be in control
\usepackage{placeins}

% natbib.sty is loaded by default. However, natbib options can be
% provided with \biboptions{...} command. Following options are
% valid:

%   round  -  round parentheses are used (default)
%   square -  square brackets are used   [option]
%   curly  -  curly braces are used      {option}
%   angle  -  angle brackets are used    <option>
%   semicolon  -  multiple citations separated by semi-colon
%   colon  - same as semicolon, an earlier confusion
%   comma  -  separated by comma
%   numbers-  selects numerical citations
%   super  -  numerical citations as superscripts
%   sort   -  sorts multiple citations according to order in ref. list
%   sort&compress   -  like sort, but also compresses numerical citations
%   compress - compresses without sorting
%
% \biboptions{comma,round}

% \biboptions{}


% Katy Huff addtions
\usepackage{xspace}
\newcommand{\Cyclus}{\textsc{Cyclus}\xspace}%
\newcommand{\Cyder}{\textsc{Cyder}\xspace}%
\usepackage{color}
\DeclareMathOperator{\erf}{erf}
\DeclareMathOperator{\erfc}{erfc}


 \journal{Progress in Nuclear Energy}

\begin{document}

\begin{frontmatter}

   % Title, authors and addresses

   % use the tnoteref command within \title for footnotes;
   % use the tnotetext command for the associated footnote;
   % use the fnref command within \author or \address for footnotes;
   % use the fntext command for the associated footnote;
   % use the corref command within \author for corresponding author footnotes;
   % use the cortext command for the associated footnote;
   % use the ead command for the email address,
   % and the form \ead[url] for the home page:
   %
   % \title{Title\tnoteref{label1}}
   % \tnotetext[label1]{}
   % \author{Name\corref{cor1}\fnref{label2}}
   % \ead{email address}
   % \ead[url]{home page}
   % \fntext[label2]{}
   % \cortext[cor1]{}
   % \address{Address\fnref{label3}}
   % \fntext[label3]{}

\title{Rapid Methods for Radionuclide Contaminant Transport in Nuclear Fuel Cycle Simulation}

   % use optional labels to link authors explicitly to addresses:
   % \author[label1,label2]{<author name>}
   % \address[label1]{<address>}
   % \address[label2]{<address>}

\author[berk]{Kathryn Huff}
  \ead{huff@berkeley.edu}
  \address[berk]{Department of Nuclear Engineering, 4155 Etcheverry Hall, MC 1730, University of California, Berkeley, CA 94720-1730}
 \end{frontmatter}

\section*{Highlights}
   \begin{itemize}
   \item \Cyder is a software library for modeling generic-geology spent nuclear fuel disposal in a nuclear fuel cycle simulator.
   \item Mathematical models of hydrologic contaminant transport were implemented in \Cyder.
   \item This suite of mathematical models is appropriate for capturing a broad range of repository concepts.
   \item \Cyder performance within the \Cyclus fuel cycle simulator is demonstrated on timescales appropriate for fuel cycle simulation.
   \item Agreement of essential physics is demonstrated between \Cyder and a more detailed stand-alone repository performance model.
   \end{itemize}

  \end{document}

  %
  % End of file `elsarticle-template-num.tex'.


