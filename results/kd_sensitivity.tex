

As the distribution coefficient $K_d$ increases, so does the retardation 
coeffcient $R_f$, according to the relation $R_f = 1+ \rho_b\frac{K_d}{\theta}$. As these two values increase, contaminants tend 
toward the solid phase. An increase in these coefficients, then, has the effect 
of limiting dissolved concentration.

In the parametric sensitivity analysis reported in \cite{huff_key_2012},
the expected inverse relationship between the retardation factor and resulting
peak annual dose was found for all elements except $^{129}I$ and $^{79}Se$. 
These two isotopes are  effectively infinitely soluble and therefore 
demonstrate no sensitivity whatsoever to a the solubility limit multiplication 
factor. In the low retardation coefficient cases, a regime is established in 
which the peak annual dose is entirely unaffected by changes in retardation 
coefficient.

For large values of retardation coefficient, the sensitivity to small changes
in the retardation coefficient increases dramatically. In that sensitive
regime, the change in peak annual dose is inversely related to the retardation
coefficient. Between these two regimes was a transition regime, in which the
$K_d$ factor ranges from $1\times10^{-5}$ to $5\times10^{0} [-]$.

It is clear from Figure \ref{fig:KdSumFactor} that
for retardation coefficients greater than a threshold, the
relationship between peak annual dose and retardation coefficient is a strong
inverse one.

\begin{figure}[ht]
\centering
\includegraphics[width=0.7\linewidth]{./results/images/Retardation_Summary_kdFactor.eps}
\caption[$K_d$ factor sensitivity in Clay GDSM]{$K_d$ factor sensitivity in 
        DOE Clay GDSM, reproduced from \cite{huff_key_2012}.
The peak annual dose due to an inventory,
$N$, of each isotope.}
\label{fig:KdSumFactor}
\end{figure}

\FloatBarrier
