%Also, for most of 5.2, you need to make the case that the single result you show
%is representative!!  I think that's easy, but needs to be done.  Essentially,
%you need to make the case that for real isotopes, the same model will be
%invoked with real parameters, so that these normalized (relative?) parameters
%are representative of all cases.

To verify the fundamental behavior of all four \Cyder radionuclide transport models at
each component interface, many transport and containment base cases were
conducted.

The simulations were conducted within the \Cyclus framework and had the
following global parameters:

\begin{itemize}
\item{A 1000 year simulation}
\item{A source facility providing one waste stream per time step}
\item{An initial capacity of five 1 kg waste streams (in most cases)}
\item{Waste form components each accepting 1 waste stream}
\item{Corresponding waste package components, one per waste form}
\item{A buffer component}
\item{A far field component}
\end{itemize}


Each feasible combination of the four models was conducted to verify
implementation of the time stepping algorithm and transport modes between
components. Among these simulations, one in which each component is represented with a Mixed
Cell model is shown in \ref{fig:mcIIIall} through \ref{fig:mcIII}.  Solubility
limitation is enabled in this case, so the system is expected to demonstrate
solubility limited transport.  To simultaneously demonstrate the behavior of
the solubility limitation, no sorption is applied, but solubility limitation is
set to 0.001 kg/m$^3$ for all isotopes.  The fixed maximum transport mode was
used between mixed cell components for speed and clarity of results.


\begin{figure}[ht]
\centering
\includegraphics[width=0.6\textwidth]{./results/images/mcIII.eps}
\caption[$^{235}U$ residence. Mixed Cell Coupled Sorption and Solubility Limitation.]{
For the MCII case in which containment is affected by solubility limitation,
($F_{d}=0.1$ for all components), $^{235}U$ travels more slowly than in the MCI case
before permanent residence in the far field component.
}
\label{fig:mcIIIall}
\begin{minipage}[b]{0.45\linewidth}

  \includegraphics[width=\textwidth]{./results/images/mcIII1.eps}
  \caption[Case MCII Waste Form Contaminants.]{
    Waste Form 5 ($F_d = 0.1$, $S_{ref} = 0.001kg/m^3$) releases material with degradation.
    }
  \label{fig:mcIIIwf5}

  \includegraphics[width=\textwidth]{./results/images/mcIII3.eps}
  \caption[Case MCII Buffer Contaminants]{
    The Buffer, component 7 ($F_d=0.1$, $S_{ref}=0.001kg/m^3$), receives and then releases material.
    }
  \label{fig:mcIIIbuff}

\end{minipage}
\hspace{0.05\linewidth}
\begin{minipage}[b]{0.45\linewidth}
  \includegraphics[width=\textwidth]{./results/images/mcIII2.eps}
  \caption[Case MCII Waste Package Contaminants.]{
    Waste Package 6 ($F_d = 0.1$, $S_{ref}=0.001kg/m^3$) receives then releases material.
    }
  \label{fig:mcIIIwp6}

  \includegraphics[width=\textwidth]{./results/images/mcIII0.eps}
  \caption[Case MCII Waste Package Contaminants.]{All material is released into
    the Far Field, component 4 ($F_d=0.0$, $S_{ref} = 0.001kg/m^3$).}
  \label{fig:mcII}


  \end{minipage}
\end{figure}



\FloatBarrier

