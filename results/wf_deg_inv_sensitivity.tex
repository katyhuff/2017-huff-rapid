In the parametric sensitivity analysis reported in \cite{huff_key_2012}, the
results showed two regimes. In the first regime, the mean of the peak annual
dose rates is directly proportional to both the mass factor (an inventory mass
multiplier) and the fractional waste form degradation rate. For some
radionuclides, attenuation occurs for high values of both parameters as the
release of radionuclides is limited by dispersion parameters. This phenomenon
can be seen in the figures below in which transition between regimes for higher
degradation rates happens at lower mass factors than transition between regimes
for lower degradation rates.

The peaks for highly soluble, non-sorbing elements such as $I$ and $Cl$
are directly proportional to mass factor for most
values of waste form degradation rates. This effect can be seen in Figures
\ref{fig:WFDegI129} and \ref{fig:WFDegCl36}.


Highly soluble and non-sorbing $^{129}I$ demonstrates a direct proportionality between dose rate and
fractional degradation rate until a turnover where other natural system
parameters dampen transport.

\begin{figure}[ht!]
\begin{minipage}[b]{0.45\linewidth}
\centering
\includegraphics[width=\linewidth]{./results/images/WFDegAndInv/I-129.eps}
\caption{$^{129}I$ waste form degradation rate sensitivity demonstrated in Clay 
        GDSM \cite{huff_key_2012}.}
\label{fig:WFDegI129}

\end{minipage}
\hspace{0.05\linewidth}
\begin{minipage}[b]{0.45\linewidth}

\includegraphics[width=\linewidth]{./results/images/WFDegAndInv/Cl-36.eps}
\caption{$^{36}Cl$ waste form degradation rate sensitivity demonstrated in Clay 
        GDSM \cite{huff_key_2012}.}
\label{fig:WFDegCl36}
\end{minipage}
\end{figure}

\FloatBarrier
