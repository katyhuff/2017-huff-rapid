\section{Results and Discussion}
\label{sec:results}

In the present work, many numerical experiments successfully verified the \Cyder software library.
Multi-component simulations demonstrated expected transport behavior and
the successful collective interaction of the modular
components in a \Cyder repository. Single-parameter sensitivity analyses
demonstrated that physics captured by the \Cyder models compare favorably to
results reported in \cite{huff_key_2012} from a more detailed existing model,
the Clay \gls{GDSM}, developed by the \gls{UFD} Campaign within the
\gls{DOE} Office of Nuclear Energy \cite{clayton_generic_2011}.

In addition to these numerical experiments, a robust unit test suite was
deployed during development to verify \Cyder software implementation.

\subsection{Multi-component Simulations}
% Many numerical experiments were conducted to verify and validate this code.

% The interaction between components is demonstrated by an example simulation

%Also, for most of 5.2, you need to make the case that the single result you show
%is representative!!  I think that's easy, but needs to be done.  Essentially,
%you need to make the case that for real isotopes, the same model will be
%invoked with real parameters, so that these normalized (relative?) parameters
%are representative of all cases.

To verify the fundamental behavior of all four \Cyder radionuclide transport models at
each component interface, many transport and containment base cases were
conducted.

The simulations were conducted within the \Cyclus framework and had the
following simulation parameters:

\begin{itemize}
\item{A 1000 year simulation}
\item{A source facility providing one waste stream per time step}
\item{An initial capacity of five 1 kg waste streams (in most cases)}
\item{No more than one waste stream object is stored per waste form}
\item{Corresponding waste package components, one per waste form}
\item{A buffer component (i.e. a bentonite clay)}
\item{A far field component (i.e. the host rock)}
\end{itemize}


Each feasible combination of the four models was conducted to verify
implementation of the time stepping algorithm and transport modes between
components. A full description of each of these verification simulations can be 
found in the dissertation \cite{huff_integrated_2013}. Among these simulations, 
one in which each component is represented with a Mixed Cell model is shown in 
\ref{fig:mcIIIall} through \ref{fig:mcIII}.  
The fixed maximum transport mode was used between mixed cell components for speed and clarity of results.

Solubility limitation is enabled 
in this case, so the system is expected to demonstrate solubility limited 
transport.  To simultaneously demonstrate the behavior of the solubility 
limitation, no sorption is applied, but solubility limitation is set to 0.001 
kg/m$^3$ for all isotopes.  Please note that the \Cyder user must currently 
provide reference solubility values for each isotope. While this offers the 
user complete control, it may be inconvenient for some users. Future extensions 
to \Cyder will include a default database for these values, perhaps through the 
\gls{PyNE} database toolkit\cite{bates_pyne_2014}. 


\begin{figure}[ht]
\centering
\includegraphics[width=0.6\textwidth]{./results/images/mcIII.eps}
\caption[$^{235}U$ residence. Mixed Cell Coupled Sorption and Solubility Limitation.]{
For the MCII case in which containment is affected by solubility limitation,
($F_{d}=0.1$ for all components), $^{235}U$ travels more slowly than in the MCI case
before permanent residence in the far field component.
}
\label{fig:mcIIIall}
\begin{minipage}[b]{0.45\linewidth}

  \includegraphics[width=\textwidth]{./results/images/mcIII1.eps}
  \caption[Case MCII Waste Form Contaminants.]{
    Waste Form 5 ($F_d = 0.1$, $S_{ref} = 0.001kg/m^3$) releases material with degradation.
    }
  \label{fig:mcIIIwf5}

  \includegraphics[width=\textwidth]{./results/images/mcIII3.eps}
  \caption[Case MCII Buffer Contaminants]{
    The Buffer, component 7 ($F_d=0.1$, $S_{ref}=0.001kg/m^3$), receives and then releases material.
    }
  \label{fig:mcIIIbuff}

\end{minipage}
\hspace{0.05\linewidth}
\begin{minipage}[b]{0.45\linewidth}
  \includegraphics[width=\textwidth]{./results/images/mcIII2.eps}
  \caption[Case MCII Waste Package Contaminants.]{
    Waste Package 6 ($F_d = 0.1$, $S_{ref}=0.001kg/m^3$) receives then releases material.
    }
  \label{fig:mcIIIwp6}

  \includegraphics[width=\textwidth]{./results/images/mcIII0.eps}
  \caption[Case MCII Waste Package Contaminants.]{All material is released into
    the Far Field, component 4 ($F_d=0.0$, $S_{ref} = 0.001kg/m^3$).}
  \label{fig:mcII}


  \end{minipage}
\end{figure}



\FloatBarrier



\subsection{Single Effect Parametric Analyses}
% Many parametric analysis were conducted to validate system responses

Each of the radionuclide contaminant transport models described in Section
\ref{sec:nuclide_models} capture different combintations of physics present in
the hydrologic contaminant transport problem. To determine how effectively
these physics were captured, single-effect simulations were conducted with
\Cyder and compared to similar analysis \cite{huff_key_2012} conducted with a
more detailed radionuclide transport model, the Clay \gls{GDSM}
\cite{clayton_generic_2011}. The Clay \gls{GDSM} was developed by the \gls{UFD}
Campaign within the \gls{DOE} Office of Nuclear Energy using the GoldSim
simulation environment \cite{golder_associates_goldsim_2010}. Hydrologic
contaminant transport in the Clay \gls{GDSM} relies on the GoldSim contaminant
transport module \cite{golder_associates_goldsim_2010-1}.

These single-effect sensitivity analyses were constructed by repeated
multi-component simulation runs conducted across the valid range for a single
parameter. To verify the behavior of a single parameter of each of the \Cyder
models, one hundred multi-component simulations were conducted, each with a
different value of that parameter.  This parametric analysis was conducted to
show that, for an arbitrary isotope, the expected dependence on that parameter
is captured. In the case of real isotopes in a full simulation, the same model
will be invoked with real parameters for each isotope. Thus, the this model
agreement is representative in all cases.

The results acheived with \Cyder were compared to the results of a similiar
parametric sensitivity analysis using the Clay \gls{GDSM} which was reported in
\cite{huff_key_2012}.

\subsubsection{Solubility Sensitivity}
To verify the behavior of the solubility limitation model in the Mixed Cell
model, for example, one hundred multi-component simulations were conducted,
each with a different reference solubility limit.
For an arbitrary isotope, the expected solubility
limitation behavior is captured and compared favorably to the Clay \gls{GDSM}
solubility limitation sensitivity results.

The results in Figure \ref{fig:SolSumFactor}, from the detailed parametric
analysis in \cite{huff_key_2012}, showed that for solubility limits below a
certain threshold, the dose releases were directly proportional to the
solubility limit, indicating that the radionuclide concentration saturated the
groundwater up to the solubility limit near the waste form.  For solubility
limits above the threshold, however, further increase to the limit had no
effect on the peak dose. This demonstrates the situation in which the
solubility limit is so high that even complete dissolution of the waste
inventory into the pore water is insufficient to reach the solubility limit.


\begin{figure}[ht]
\begin{center}
\includegraphics[width=0.7\linewidth]{./results/images/Solubility_Summary_SolFactor.eps}
\caption[Solubility factor sensitivity in the Clay GDSM model]{Solubility
        factor sensitivity in the DOE Clay GDSM, reproduced from 
        \cite{huff_key_2012}. The peak annual dose due to an inventory, $N$, of each
isotope. This result was achieved with a parametric analysis using a detailed
        model of a generic clay repository.}
\label{fig:SolSumFactor}
\end{center}
\end{figure}

\begin{figure}[ht]
\begin{center}
\includegraphics[width=0.7\linewidth]{./results/images/sol.eps}
\caption[Solubility Sensitivity in the Mixed Cell Model]{Sensitivity demonstration of solubility limitation in \Cyder for an arbitrary isotope assigned a variable solubility limit.}
\label{fig:sol_result}
\end{center}
\end{figure}


In the corresponding parametric analysis of \Cyder performance, it was shown that the
solubility sensitivity behavior closely matched that of the \gls{GDSM}
sensitivity behaviors. Specifically, in Figure \ref{fig:sol_result}, a marked 
transition to the solubility-limited regime
is seen where the solubility limit exceeds the point at which it limits
movement. For increased solubility limits, release remains constant, as
expected.

In both \Cyder and the more detailed Clay \gls{GDSM}, for solubility constants
lower than the saturation threshold, the transport regime is solubility
limited and the relationship between peak annual dose and solubility limit is
strong.  Above the threshold, the transport regime is inventory limited
instead.

%\begin{figure}[ht]
%\begin{center}
%\includegraphics[width=0.7\linewidth]{./results/images/Solubility_Summary_Sol.eps}
%\caption[Solubility limit sensitivity in GDSM Clay model]{Solubility limit sensitivity. The peak annual dose due to an inventory,
%$N$, of each isotope.}
%\label{fig:SolSum}
%\end{center}
%\end{figure}


\FloatBarrier
\subsubsection{Sorption Sensitivity}



As the distribution coefficient $K_d$ increases, so does the retardation 
coeffcient $R_f$, according to the relation $R_f = 1+ \rho_b\frac{K_d}{\theta}$. As these two values increase, contaminants tend 
toward the solid phase. An increase in these coefficients, then, has the effect 
of limiting dissolved concentration.

In the parametric sensitivity analysis reported in \cite{huff_key_2012},
the expected inverse relationship between the retardation factor and resulting
peak annual dose was found for all elements except $^{129}I$ and $^{79}Se$. 
These two isotopes are  effectively infinitely soluble and therefore 
demonstrate no sensitivity whatsoever to a the solubility limit multiplication 
factor. In the low retardation coefficient cases, a regime is established in 
which the peak annual dose is entirely unaffected by changes in retardation 
coefficient.

For large values of retardation coefficient, the sensitivity to small changes
in the retardation coefficient increases dramatically. In that sensitive
regime, the change in peak annual dose is inversely related to the retardation
coefficient. Between these two regimes was a transition regime, in which the
$K_d$ factor ranges from $1\times10^{-5}$ to $5\times10^{0} [-]$.

It is clear from Figure \ref{fig:KdSumFactor} that
for retardation coefficients greater than a threshold, the
relationship between peak annual dose and retardation coefficient is a strong
inverse one.

\begin{figure}[ht]
\centering
\includegraphics[width=0.7\linewidth]{./results/images/Retardation_Summary_kdFactor.eps}
\caption[$K_d$ factor sensitivity in Clay GDSM]{$K_d$ factor sensitivity in 
        DOE Clay GDSM, reproduced from \cite{huff_key_2012}.
The peak annual dose due to an inventory,
$N$, of each isotope.}
\label{fig:KdSumFactor}
\end{figure}

\FloatBarrier


In the parametric analysis of \Cyder performance, it was shown that sorption
sensitivity behavior closely matched that of the \gls{GDSM} sensitivity
behaviors. Specifically, in Figure \ref{fig:kd_result}, increasing the retardation
coefficient results in a smooth but dramatic turnover.

\begin{figure}[ht]
\centering
\includegraphics[width=0.7\linewidth]{./results/images/kd.eps}
\caption[$K_d$ sensitivity in the Mixed Cell Model]{$K_d$ sensitivity in the
\Cyder tool for an arbitrary isotope assigned a variable $K_d$ coefficient.}
\label{fig:kd_result}
\end{figure}


\FloatBarrier
\subsubsection{Waste Form Degradation Rate Sensitivity}
In the parametric sensitivity analysis reported in \cite{huff_key_2012}, the
results showed two regimes. In the first regime, the mean of the peak annual
dose rates is directly proportional to both the mass factor (an inventory mass
multiplier) and the fractional waste form degradation rate. For some
radionuclides, attenuation occurs for high values of both parameters as the
release of radionuclides is limited by dispersion parameters. This phenomenon
can be seen in the figures below in which transition between regimes for higher
degradation rates happens at lower mass factors than transition between regimes
for lower degradation rates.

The peaks for highly soluble, non-sorbing elements such as $I$ and $Cl$
are directly proportional to mass factor for most
values of waste form degradation rates. This effect can be seen in Figures
\ref{fig:WFDegI129} and \ref{fig:WFDegCl36}.


Highly soluble and non-sorbing $^{129}I$ demonstrates a direct proportionality between dose rate and
fractional degradation rate until a turnover where other natural system
parameters dampen transport.

\begin{figure}[ht!]
\begin{minipage}[b]{0.45\linewidth}
\centering
\includegraphics[width=\linewidth]{./results/images/WFDegAndInv/I-129.eps}
\caption{$^{129}I$ waste form degradation rate sensitivity demonstrated in Clay 
        GDSM \cite{huff_key_2012}.}
\label{fig:WFDegI129}

\end{minipage}
\hspace{0.05\linewidth}
\begin{minipage}[b]{0.45\linewidth}

\includegraphics[width=\linewidth]{./results/images/WFDegAndInv/Cl-36.eps}
\caption{$^{36}Cl$ waste form degradation rate sensitivity demonstrated in Clay 
        GDSM \cite{huff_key_2012}.}
\label{fig:WFDegCl36}
\end{minipage}
\end{figure}

\FloatBarrier

\input{./results/wf_deg_inv_results}

\FloatBarrier


\subsection{Significance}
% The existence of this code enables dynamic analysis of repository performance
% during fuel cycle simulation.

This work has provided a flexible software library for rapid medium-fidelity
calculation of generic repository performance in the context of fuel cycle
analysis.  Capable of hydrologic contaminant transport and integration within a
fuel cycle simulation library, \Cyder is the first of its kind.

In this work, modeling methods for geologic radioactive waste disposal
performance analysis were described as was their implementation in the \Cyder
repository performance library. The application programming interface to this
software library is intentionally general, facilitating the incorporation of
the models presented here within external software tools.

\Cyder performance within the \Cyclus fuel cycle simulator and agreement
between Cyder and a more detailed stand-alone model were also demonstrated.
\Cyder methods make a strategic tradeoff between speed and fidelity, capturing
essential physics when computing back-end nuclear fuel cycle metrics. The
result is a library of medium-fidelity hydrologic contaminant transport models
within a disposal facility simulation framework appropriate for use in dynamic
nuclear fuel cycle simulators.

The \Cyder source code is freely available to interested researchers and
potential model developers \cite{huff_cyder_2013}.  In addition to the source
code and supporting publications, the \Cyder library is well commented and
produces clickable, browsable automated documentation with each build. That
documentation is also available online.

Finally, this work contributes to an expanding ecosystem of computational
models available for use with the \Cyclus fuel cycle simulator. This hydrologic
nuclide transport library, by virtue of its capability to modularly integrate
with the \Cyclus fuel cycle simulator has laid the foundation for integrated
disposal option analysis in the context of fuel cycle options.
